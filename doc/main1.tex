% !TeX spellcheck = ru_RU

\documentclass[a5paper,12pt]{article}
\usepackage[cache=true]{minted}
\usepackage{polyglossia}
\setmainlanguage{russian}
\let\cyrillicfonttt\monofamily
\usepackage{comment}
\usepackage{stmaryrd}

\usepackage[ left=1cm
           , right=2cm
           , top=1cm
           , bottom=1.5cm
           ]{geometry}
\usepackage{amssymb,amsmath,amsthm,amsfonts} 

\usepackage{fontspec}
% \DeclareMathSizes{22}{30}{24}{20}

%\usefonttheme{professionalfonts}
\defaultfontfeatures{Ligatures={TeX}}
%\setmainfont[Scale=1.5]{Times New Roman}
%\setmainfont{Latin Modern Roman}
\setmainfont [ Scale=1]{CMU Serif Roman}
\setsansfont[Scale=1]{CMU Sans Serif}

%\setmonofont[ BoldFont=lmmonolt10-bold.otf
%			, ItalicFont=lmmono10-italic.otf
%			, BoldItalicFont=lmmonoproplt10-boldoblique.otf
%			, Scale=1.5
%]{lmmono9-regular.otf}
%\setmonofont[Scale=1.5]{CMU Typewriter Text}
\setmonofont{CMU Typewriter Text}

\usepackage{unicode-math}
\setmathfont{Latin Modern Math}[Scale=1]
\newcommand*{\arr}{\ensuremath{\rightarrow}}

% Doesn't work?
\renewcommand{\epsilon}{\ensuremath{\varepsilon}}
%\renewcommand{\sigma}{\ensuremath{\varsigma}}
\newcommand{\inbr}[1]{\left<{#1}\right>}
\newcommand{\ruleno}[1]{\mbox{[\textsc{#1}]}}
\newcommand{\bigslant}[2]{{\raisebox{.2em}{$#1$}\left/\raisebox{-.2em}{$#2$}\right.}}
\newcommand{\sem}[1]{\llbracket #1 \rrbracket}
\newcommand{\ir}{\ensuremath{\mathcal{I\!R}}}
\begin{document}

%\begin{minted}[mathescape, escapeinside=??]{ocaml }  
%val encode: term -> heap -> term  (* ??? *)
%
%let encode term ?$\sigma$? = 
%  gmap term (object 
%                method LI h x ?$\arr$? find h x ?$\sigma$? 
%                method Call f ?$\arr$? encode (result_of f) ?$\sigma$? 
%            end)
%
%let find heap x ?$\sigma$? =
%  match heap with 
%  | Defined hs  ?$\arr$? find_defined hs x ?$\sigma$? 
%  | Merge [<?$g_i$?, ?$h_i$?>; ...] ?$\arr$? 
%     UNION [ <encode ?$g_i$? ?$\sigma$?, find ?$h_i$? x ?$\sigma$?>; ...]
%  | RecApp f  ?$\arr$? find (effect_of f) x ?$\sigma$?
%  | h1 ?$\circ$? h2   ?$\arr$? find h2 x (fun x -> find h1 x ?$\sigma$?)
%\end{minted}

\section{Дизъюнктный паттерн мэтчинг}

$$
\mathcal{P = \_ \mid CP^*}
$$

$$
\mathcal{V = CV^*}
$$

$$
\mathcal{S}(p_1,\dots,p_n) = \lambda v \rightarrow \sem{v; p_1,\dots,p_n}
$$

\begin{figure}
  \[
  \begin{array}{cr}
    \inbr{v; \_} \rightarrow 1 & \ruleno{WildCard}\\[2mm]
%    \dfrac{R_i^{k_i}=\lambda\,x_1\dots x_{k_i}\,.\,g}{\inbr{R_i^{k_i}\,(t_1,\dots,t_{k_i}),\sigma,n} \xrightarrow{\circ} \inbr{g\,[\bigslant{t_1}{x_1}\dots\bigslant{t_{k_i}}{x_{k_i}}], \sigma, n}} & \ruleno{Invoke}\\[5mm]
    \dfrac{\inbr{v,p_i}}{\sem{v;p_1,\dots, p_n}} & \ruleno{rule1}\\[5mm]
    \dfrac{\inbr{v_1,p_1},\dots,\inbr{v_n,p_n}}{\inbr{C\ v_1\dots v_n; C\ p_1,\dots, p_n}} & \ruleno{rule2}
%    \\[5mm]
%    \inbr{t_1 \equiv t_2, \sigma, n} \xrightarrow{\circ} \Diamond , \, \, \nexists\; mgu\,(t_1, t_2, \sigma) &\ruleno{UnifyFail} \\[2mm]
%    \inbr{t_1 \equiv t_2, \sigma, n} \xrightarrow{(mgu\,(t_1, t_2, \sigma),\, n)} \Diamond & \ruleno{UnifySuccess} \\[2mm]
%    \inbr{g_1 \lor g_2, \sigma, n} \xrightarrow{\circ} \inbr{g_1, \sigma, n} \oplus \inbr{g_2, \sigma, n} & \ruleno{Disj} \\[2mm]
%    \inbr{g_1 \land g_2, \sigma, n} \xrightarrow{\circ} \inbr{ g_1, \sigma, n} \otimes g_2 & \ruleno{Conj} \\[2mm]
%    \inbr{\mbox{\texttt|fresh|}\, x\, .\, g, \sigma, n} \xrightarrow{\circ} \inbr{g\,[\bigslant{\alpha_{n + 1}}{x}], \sigma, n + 1} & \ruleno{Fresh}\\[2mm]
%    \dfrac{R_i^{k_i}=\lambda\,x_1\dots x_{k_i}\,.\,g}{\inbr{R_i^{k_i}\,(t_1,\dots,t_{k_i}),\sigma,n} \xrightarrow{\circ} \inbr{g\,[\bigslant{t_1}{x_1}\dots\bigslant{t_{k_i}}{x_{k_i}}], \sigma, n}} & \ruleno{Invoke}\\[5mm]
%    \dfrac{s_1 \xrightarrow{\circ} \Diamond}{(s_1 \oplus s_2) \xrightarrow{\circ} s_2} & \ruleno{DisjStop}\\[5mm]
%    \dfrac{s_1 \xrightarrow{r} \Diamond}{(s_1 \oplus s_2) \xrightarrow{r} s_2} & \ruleno{DisjStopAns}\\[5mm]
%    \dfrac{s \xrightarrow{\circ} \Diamond}{(s \otimes g) \xrightarrow{\circ} \Diamond} &\ruleno{ConjStop}\\[5mm]
%    \dfrac{s \xrightarrow{(\sigma, n)} \Diamond}{(s \otimes g) \xrightarrow{\circ} \inbr{g, \sigma, n}}  & \ruleno{ConjStopAns}\\[5mm]
%    \dfrac{s_1 \xrightarrow{\circ} s'_1}{(s_1 \oplus s_2) \xrightarrow{\circ} (s_2 \oplus s'_1)} &\ruleno{DisjStep}\\[5mm]
%    \dfrac{s_1 \xrightarrow{r} s'_1}{(s_1 \oplus s_2) \xrightarrow{r} (s_2 \oplus s'_1)} &\ruleno{DisjStepAns}\\[5mm]
%    \dfrac{s \xrightarrow{\circ} s'}{(s \otimes g) \xrightarrow{\circ} (s' \otimes g)} &\ruleno{ConjStep}\\[5mm]
%    \dfrac{s \xrightarrow{(\sigma, n)} s'}{(s \otimes g) \xrightarrow{\circ} (\inbr{g, \sigma, n} \oplus (s' \otimes g))} & \ruleno{ConjStepAns} 
  \end{array}
  \]
  \caption{Non-deterministic semantioc of disjunctive pattern matching}
%  \label{lts}
\end{figure}


\section{Промежуточное представление}
\begin{align*}
\mathcal{C} =&\; \{ C_1^{k_1}, \dots, C_n^{k_n} \} \\
\mathcal{V} =&\;  \mathcal{C}\ \mathcal{V}^*\\
\mathcal{M} =&\;  \mathcal{S} \\
          \mid\; &\; \text{\texttt{Field}}\  \mathcal{M}\times  \mathbb{N}\\
\mathcal{P} =&\;  \mathcal{WC} \\
          \mid\; &\; \text{\texttt{Var}}\  \mathcal{Name}\\
          \mid\; &\; \text{\texttt{PConstructor}}\  \mathcal{C}\times  \mathcal{P}^*\\
\ir  =&\; \mathcal{S} \\
           \mid\; &\;\text{\texttt{Int}}\  \mathbb{N} \\
%           \mid\; &\;\text{\texttt{Expr}}\  \mathcal{V} \\
           \mid\; &\; \text{\texttt{IfTag}}\; \mathcal{C}\times \mathcal{M}\times \ir\times \ir\\
           \mid\; &\; \text{\texttt{IfGuard}}\ \mathbb{N}\times (\mathcal{Name}\times \mathcal{M})^*\times \ir\times \ir\\
\mathcal{Clause} =&\;  \mathcal{P} \times \mathbb{N}? \times \ir \\           
\end{align*}

Сопоставление с образцом производится для значений вида $\mathcal{V}$. Среди паттернов разрешены wildcards, переменной и паттерны-констукторы с аргументами. В клаузхах у паттернов разрешено ставить ноль или одно охранное выражение, которые нумеруются натуральными числами $\mathbb{N}$. В правой части клауз стоит произвольный код вида $\ir$.

\section{Интерпретатор паттернов}
Чтобы написать интепретатор (реляционный или нет) сопоставления с образцом, который на выходе выдает значение вида $\ir$, на входе необходимо иметь
\begin{enumerate}
\item конкретное выражение $v\in \mathcal{V}$ (scrutinee)
\item клаузы, возможно c охранными выражениями, представленными числами из $\mathbb{N}$;
\item конкретный результат на случай, если ни один паттерн не подошел
\item функцию-интерпретатор охранных выражений, которая принимает номер и сопоставления переменных мэтчинга в выражения из $\mathcal{V}$
\end{enumerate}

Такой интерпретатор будем обозначать $\sem{s;p_1\dots,p_n}$.



\section{Интерпретатор $\ir$}
В язык $\ir$ не введены в явном виде термы из языка $\mathcal{V}$, чтобы сделать язык менее богатым и, следовательно, более поддающемся перебору. Поэтому конструкторы \texttt{IfTag} и \texttt{IfGuard} содержат в себе не выражения $\mathcal{V}$, а срезы исходного scrutinee $\mathcal{M}$.

Для написания интепретатора $eval_{\ir}$ требуется 
\begin{itemize}
\item Программа, которая будет интерпретироваться
\item Выражение, которое будет разбираться (scrutinee)
\item Функция-интепретатор исполнения охранного выражения c номером, на частичном отображении некоторых имен в подвыражения scrutinee.
\end{itemize}

\section{Задача синтеза}

Дана программа, производящие сопоставление с образцом какое-то значение 
scrutinee конкретного типа с помощью конкретных паттернов. В паттернах разрешены конструкторы, wildcards и охранные выражения. В правых частях стоит слово языка $\ir$, обозначающее номер ветки. Необходимо по данной программе построить программу на языке $\ir$, которая показывает такое же поведение, и оптимальная с точки зрения количества кода.

Таким образом ставится задачи синтеза программы $pr\in~\ir$, которая для произвольных scrutinee $s$  и конкретных паттернов с охранными выражениями $p_1\dots,p_n$ ведет себя так же, как и исходная программа
$$
\forall s.\; eval_{\ir}\; s\; pr = \sem{s;p_1\dots,p_n}
$$
или в реляционном синтаксисе

$$
\forall s\; \exists res: \left((eval^o_{\ir}\; s \; pr \; res) \land ( res = \sem{s;p_1\dots,p_n}) \right)
$$

В данной формуле стоит универсальная квантификация по возможным $s\in\mathcal{S}$, которая затрудняет перебор. От неё можно избавиться (раздел \ref{genexamples}), сведя её к перебору по \emph{конечному} набору примеров $\mathcal{ExS}$:

$$
\forall s \; (s\in\mathcal{ExS})\land \left(\exists res \left((eval^o_{\ir}\; s \; pr \; res) \land ( res = \sem{s;p_1\dots,p_n}) \right)\right)
$$

\section{Получение конечного числа примеров}
\label{genexamples}

Идея заключается в том, что для конкретных паттернов нам известна максимальная высота паттернов разбираемого выражения: для примеров с большей высотой некоторые подтермы не будут затронуты сопоставлением с образцом и по сути могут иметь произвольное значение. Таким образцом, если тип населен бесконечным числом жителей, то достаточно проверить только на жителях конечной высоты.

Для алгебраических типов данных все примеры конечной высоты генерируются за конечное время. Обобщенные (GADT) алгебраические типы данных могут быть не населены, и, в общем случае, процесс построения конечного числа примеров может не завершиться.

Население n-параметрического (G)ADT осуществляется реляцией с $(1+2n)$ аргументами: один для результата и на каждый параметр $\alpha$ по реляции, населяющей тип $\alpha$ и информации $info_{\alpha}$ о конкретной инстанциации этого типового параметра.

Информация об инстанции хранится в типе $ginfo$, погруженным в реляционный домен. Значения такого типа хранят в себе имя типа и список аргументов, обозначающий с какими конкретными типами он был инстанциирован. Если типовый параметр -- это полиморфная типовая переменная, то вместо информации мы будем использовать свежую переменную реляционного программирования.

\begin{minted}{ocaml}
type ginfo = Info of string * ginfo list
\end{minted}

Не дописано. Пример.

\begin{minted}{ocaml}
type _ expr =
  | Int  : int -> int expr
  | Bool : bool -> bool expr
  | If   : bool expr * 'a expr * 'a expr -> 'a expr
  
let rec inhabit_expr = fun inh_arg arg_desc r ->
  conde
    [ fresh (i)
        (Info.int === arg_desc)
        (r === int i)
        (inhabit_int i)
    ; fresh (b)
        (Info.bool === arg_desc)
        (r === bool b)
        (inhabit_bool b)
    ; fresh (c th el)
        (if_ c th el === r)
        (inhabit_expr inhabit_bool Info.bool c)
        (inhabit_expr inh_arg arg_desc th)
        (inhabit_expr inh_arg arg_desc el)
    ]
\end{minted}


\section{Неведомая херня, которую я пытаюсь запилить}
Входной пример
\begin{minted}{ocaml}
match (... : bool * bool) with
| pair (true, _) -> 0
| pair (_, true) -> 1
| pair (_, _) -> 2
\end{minted}
Правильный ответ
\begin{minted}{ocaml}
switch S[0] with
| true -> 0
| _ -> (switch S[1] with
        | true -> 1
        | _ -> 2)
\end{minted}
Примеры с вайлдкардами: номер + описание адекватного скрутини + список полей, куда можно заглядывать
\begin{minted}{ocaml}
[ (0, (fun q -> (q === pair true_ __))
    , GroundField.[ field0 ])
; (1, (fun q ->
         fresh () 
           (q =/= pair true_ __) 
           (q === pair __ true_))
    , GroundField.[ field1 ] )
; (2, ...)
]
\end{minted}
Будем пытаться запустить программу-правильный-ответ на примере №1.
\begin{itemize}
\item Теперь нам надо проверить, 
\begin{itemize}
\item Эта ветка подходит нашему скрутини
\item Этот кусок скрутини можно мэтчить
\end{itemize}
\item Имеем варианты
\begin{enumerate}
\item (да, да) -- рекурсивно исполняем rhs 
\item (да, нет) -- все правые части должны либо не давать ни одного ответа, или все вместе выдавать один и тот же  
\item (нет, да) -- разбираем следующую ветку
\item (нет, нет) -- разбираем следующую ветку
\end{enumerate}
\item пытаемся смэтчить \verb=S[0]=, оно нам не подходит и вообще запрещено. Поэтому 4й случай.
\end{itemize}
\end{document}
