\documentclass[submission,copyright,creativecommons]{eptcs}
%\providecommand{\event}{SOS 2007} % Name of the event you are submitting to

\usepackage{breakurl}             % Not needed if you use pdflatex only.
\usepackage{underscore}           % Only needed if you use pdflatex.
\usepackage{amssymb}
\usepackage{listings}
\usepackage{indentfirst}
\usepackage{verbatim}
\usepackage{amsmath, amsthm, amssymb}
\usepackage{graphicx}
\usepackage{url}
\usepackage{hyperref}
\usepackage{xspace}
\usepackage{placeins}

\lstdefinelanguage{scheme}{
keywords={define, conde, fresh},
sensitive=true,
%basicstyle=\small,
commentstyle=\scriptsize\rmfamily,
keywordstyle=\ttfamily\underbar,
identifierstyle=\ttfamily,
basewidth={0.5em,0.5em},
columns=fixed,
fontadjust=true,
literate={==}{{$\equiv$}}1
}

\lstdefinelanguage{ocaml}{
keywords={fresh, conde, let, begin, end, in, match, type, and, fun, function, try, with, when, class,
object, method, of, rec, repeat, until, while, not, do, done, as, val, inherit,
new, module, sig, deriving, datatype, struct, if, then, else, open, private, virtual, include, @type},
sensitive=true,
commentstyle=\small\itshape\ttfamily,
keywordstyle=\ttfamily\underbar,
identifierstyle=\ttfamily,
basewidth={0.5em,0.5em},
columns=fixed,
fontadjust=true,
literate={->}{{$\to\;\;$}}3 {===}{{$\equiv$}}3 {=/=}{{$\not\equiv$}}3 {|>}{{$\triangleright$}}3,
morecomment=[s]{(*}{*)}
}

\lstset{
mathescape=true,
identifierstyle=\ttfamily,
keywordstyle=\bfseries,
commentstyle=\scriptsize\rmfamily,
basewidth={0.5em,0.5em},
fontadjust=true,
language=ocaml
}

\sloppy

\newcommand{\miniKanren}{miniKanren\xspace}

\title{Compiling Pattern Matching via Relational Programming}
\author{Dmitry Kosarev
\institute{Saint Petersburg State University\\ Saint Petersburg, Russia}
\email{Dmitrii.Kosarev@protonmail.ch}
\and
Dmitry Boulytchev
\institute{Saint Petersburg State University\\ Saint Petersburg, Russia}
\email{dboulytchev@math.spbu.ru}
}
\def\titlerunning{Compiling Pattern Matching via Relational Programming}
\def\authorrunning{Dmitry Kosarev, Dmitry Boulytchev}
\begin{document}
\maketitle

\begin{abstract}
We present an implementation of the relational programming language \miniKanren as a set
of combinators and syntax extensions for OCaml. The key feature of our approach is
\emph{polymorphic unification}, which can be used to unify data structures of arbitrary types.
In addition we provide a useful generic programming pattern to systematically develop relational
specifications in a typed manner, and address the problem of integration of relational subsystems into
functional applications.
\end{abstract}

\section{Introduction}
\label{intro}

Relational programming~\cite{TRS} is an attractive technique, based on the idea
of constructing programs as relations.  As a result, relational programs can be
``queried'' in various ``directions'', making it possible, for example, to simulate
reversed execution. Apart from being interesting from a purely theoretical standpoint,
this approach may have a practical value: some problems look much simpler
when considered as queries to some relational specification~\cite{WillThesis}. There are a
number of appealing examples confirming this observation: a type checker
for simply typed lambda calculus (and, at the same time, a type inferencer and solver
for the inhabitation problem), an interpreter (capable of generating ``quines''~---
programs producing themselves as a result)~\cite{Untagged}, list sorting (capable of
producing all permutations), etc.

Many logic programming languages, such as Prolog, Mercury~\cite{MercuryFirstPaper},
or Curry~\cite{CurryFirstPaper} to some extent
can be considered relational. We have chosen \miniKanren\footnote{\url{http://minikanren.org}}
as a model language, because it was specifically designed as a relational DSL, embedded in Scheme/Racket.
Being rather a minimalistic language, which can be implemented with just a few data structures and
combinators~\cite{MicroKanren, MuKanrenNew}, \miniKanren found its way into dozens of host languages, including Scala, Haskell and Standard ML.
The paradigm behind \miniKanren can be described as ``lightweight logic programming''\footnote{An in-depth comparison of \miniKanren
and Prolog can be found here: \url{http://minikanren.org/minikanren-and-prolog.html}}.

This paper addresses the problem of embedding \miniKanren into OCaml\footnote{\url{http://ocaml.org}}~--- a statically-typed functional language with
a rich type system. A statically-typed implementation would bring us a number of benefits. First, as always,
we expect typing to provide a certain level of correctness guarantees, ruling out some pathological programs, which
otherwise would provide pathological results. In the context of relational programming, however, typing would additionally
help us to interpret the results of queries. Often an answer to a relational query contains a number of
free variables, which are supposed to take arbitrary values. In the typed case these variables become typed,
facilitating the understanding of the answers, especially those with multiple free variables. Next, a number of \miniKanren
applications require additional constraints to be implemented. In the untyped setting, when everything can be anything,
some symbols or data structures tend to percolate into undesirable contexts~\cite{Untagged}. In order to prevent this from happening, some
auxiliary constraints (``\lstinline{absent$^o$}'', ``\lstinline{symbol$^o$}'', etc.) were introduced. These constraints play a role
of a weak dynamic type system, cutting undesirable answers out at runtime. Conversely, in a typed language this work can be
entrusted to the type checker (at the price of enforcing an end user to write properly typed specifications), not only improving the
performance of the system but also reducing the number of constraints which have to be implemented. Finally, it is rather natural
to expect better performance of a statically-typed implementation.

We present an implementation of a set of relational combinators and syntax extensions for
OCaml\footnote{The source code of our implementation is accessible from \url{https://github.com/dboulytchev/OCanren}.},
which, technically speaking, corresponds to $\mu$Kanren~\cite{MicroKanren} with disequality
constraints~\cite{CKanren}. The contribution of our work is as follows:

\begin{enumerate}
\item Our embedding allows an end user to enjoy strong static typing and type inference in relational
specifications; in particular, all type errors are reported at compile time and the types for
all logical variables are inferred from the context.

\item Our implementation is based on the \emph{polymorphic unification}, which, like the polymorphic comparison,
can be used for arbitrary types. The implementation of polymorphic unification uses unsafe features and
relies on the intrinsic knowledge of the runtime representation of values; we show, however, that this does not
compromise type safety.

\item We describe a uniform and scalable pattern for using types for relational programming, which
helps in converting typed data to and from the relational domain. With this pattern, only one
generic feature (``\lstinline{Functor}'') is needed, and thus virtually any generic
framework for OCaml can be used. Although being rather a pragmatic observation, this pattern, we
believe, would lead to a more regular and easy way to maintain relational specifications.

\item We provide a simplified way to integrate relational and functional code. Our approach utilizes
a well-known pattern~\cite{Unparsing, DoWeNeed} for variadic function implementation and makes it
possible to hide the reification of the answers phase from an end user.
\end{enumerate}

The rest of the paper is organized as follows: in the next section we provide a short overview of the related
works. Then we briefly introduce \miniKanren in
its original form to establish some notions; we do not intend to describe the language in its full bloom (interested readers can
refer to~\cite{TRS}). In Section~\ref{sec:goals} we describe some basic constructs behind a \miniKanren implementation, this time
in OCaml. In Section~\ref{sec:unification} we discuss polymorphic unification, and show that unification with
triangular substitution respects typing. Then we present our approach to handle user-defined types by injecting them
into the logic domain, and describe a convenient generic programming pattern, which can be used to implement the conversions from/to logic
domain. We also consider a simple approach and a more elaborate and efficient tagless variant (see Section~\ref{sec:injection}).
Section~\ref{sec:reification} describes top-level primitives and addresses the problem of relational and functional code integration.
Then, in Section~\ref{sec:examples} we present a set of relational examples, written with the aid of our
library. Section~\ref{sec:evaluation} contains the results of a performance evaluation and a comparison of our implementations
with existing implementation for Scheme. The final section concludes.

The authors would like to express a deep gratitude to the anonymous rewievers for their numerous constructive comments, Michael Ballantyne, Greg Rosenblatt, 
and the other attendees of the miniKanren Google Hangouts, who helped the authors in understanding the subtleties of the original miniKanren
implementation, Daniel Friedman for his remarks on the preliminary version of the paper, and William Byrd for all his help and support, which cannot be
overestimated.


\section{Related Works}
\label{sec:relworks}

There is a predictable difficulty in implementing \miniKanren for a strongly typed language.
Designed in the metaprogramming-friendly and dynamically typed realm of Scheme/Racket, the original
\miniKanren implementation pays very little attention to what has a significant importance in (specifically)
ML or Haskell. In particular, one of the capstone constructs of \miniKanren~--- unification~--- has to work for
different data structures, which may have types different beyond parametricity.

There are a few ways to overcome this problem. The first one is simply to follow the untyped paradigm and
provide unification for some concrete type, rich enough to represent any reasonable data structures.
Some Haskell \miniKanren libraries\footnote{\url{https://github.com/JaimieMurdock/HK}, \url{https://github.com/rntz/ukanren}}
as well as the previous OCaml implementation\footnote{\url{https://github.com/lightyang/minikanren-ocaml}} take this approach.
As a result, the original implementation can be retold with all its elegance; the relational specifications, however,
become weakly typed. A similar approach was taken in early works on embedding Prolog into Haskell~\cite{PrologInHaskell}.

Another approach is to utilize \emph{ad hoc} polymorphism and provide a type-specific unification for each ``interesting'' type.
Some \miniKanren implementations, such as Molog\footnote{\url{https://github.com/acfoltzer/Molog}} and
MiniKanrenT\footnote{\url{https://github.com/jvranish/MiniKanrenT}}, both for Haskell, can be mentioned as examples.
While preserving strong typing, this approach requires a lot of ``boilerplate''
code to be written, so some automation --- for example, using 
Template Haskell~\cite{SheardTMH}~---
is desirable. In~\cite{TypedLogicalVariables} a separate type class was introduced to both perform the unification
and detect free logical variables in end-user data structures. The requirement for end user to provide a way to represent
logical variables in custom data structures looks superfluous for us since these logical variables would require proper
handling in the rest of the code outside the logical programming subsystem.

There is, actually, another potential approach, but we do not know if anybody has tried
it: implementing unification for a generic representation of types as sum-of-products and fixpoints of
functors~\cite{InstantGenerics, ALaCarte}. Thus, unification would work for any type for which a representation
is provided. We believe that implementing this representation would require less boilerplate code to be written.

As follows from this exposition, a typed embedding of \miniKanren in OCaml can be done with
a combination of datatype-generic programming~\cite{DGP} and \emph{ad hoc} polymorphism. There are 
a number of generic frameworks for OCaml (for example,~\cite{Deriving}). On the other hand, the support
for \emph{ad hoc} polymorphism in OCaml is weak; there is nothing comparable in power to Haskell
type classes, and even though sometimes the object-oriented layer of the language can be used to mimic
desirable behavior, the result, as a rule, is far from satisfactory. Existing proposals for \emph{ad hoc} polymorphism (for example,
modular implicits~\cite{Implicits}) require patching the compiler, which we want to avoid. Therefore, we 
take a different approach, implementing polymorphic unification once and for all logical types~--- a purely \emph{ad hoc} 
approach, since the features which would provide a less \emph{ad hoc} solution are not yet well integrated into the language. To deal
with user-defined types in the relational subsystem, we propose to use their logical representations (see Section~\ref{sec:injection}), 
which free an end user from the burden of maintaining logical variables, and we use generic programming to build conversions from and to logical
representations in a systematic manner.




%\begin{figure}[t]
%\centering
%\includegraphics{graph2.pdf}
%\caption{The Second Set of Benchmarks}
%\label{eval:second}
%\end{figure}

\section{Performance Evaluation}
\label{sec:evaluation}

One of our initial goals was to evaluate what performance impact would choosing OCaml as a host language makes. In addition we spent some
effort in order to implement \miniKanren in an efficient, tagless manner, and, of course, the outcome of this decision also has to be
measured. For comparison we took faster-miniKanren\footnote{\url{https://github.com/webyrd/faster-miniKanren}}~--- a full-fledged
\miniKanren implementation for Scheme/Racket. It turned out that faster-miniKanren implements a number of optimizations~\cite{WillThesis, Optimizations}
to speed up the search; moreover, the search order in our implementation initially was a little bit different. In order to make the comparison
fair, we additionally implemented all these optimizations and adjusted the search order to exactly coincide with
what faster-miniKanren does.

%\begin{figure}[t]
%\centering
%\includegraphics[scale=0.4]{graph.png}
%\caption{The Results of the Performance Evaluation}
%\label{eval}
%\end{figure}

\FloatBarrier

For the set of benchmarks we took the following problems:

\begin{itemize}
\item \textbf{pow, logo}~--- exponentiation and logarithm for integers in binary form. The concrete tests relationally computed
$3^5$ (which is 243) and $log_3 243$ (which is, conversely, 5). The implementaion was adopted from~\cite{KiselyovArithm}.
\item \textbf{quines, twines, trines}~--- self/co-evaluating program synthesis problems from~\cite{Untagged}. The
concrete tests took the first 100, 15 and 2 answers for these problems respectively.
\end{itemize}

%Since the last bundle of benchmarks uses disequality constraints (and, hence, $\mu$Kanren is ruled out) we
%split all benchmarks into two sets.

The evaluation was performed on a desktop computer with Intel Core i7-4790K CPU @ 4.00GHz processor and 16GB of memory.
For OCanren \mbox{ocaml-4.04.0+frame_pointer+flambda} was used, for faster-miniKanren~--- Chez~Scheme~9.4.1.
All benchmarks were executed in the natively compiled mode ten times, then average user time was taken. The results of the evaluation
are shown on Figure~\ref{eval}. The whole evaluation repository with all scripts and detailed description is accessible
from \lstinline{GitHub}\footnote{\url{https://github.com/Kakadu/ocanren-perf}}.

The first conclusion, which is rather easy to derive from the results, is that the tagless approach indeed matters. Our initial
implementation did not show essential speedup in comparison even with $\mu$Kanren (and was even \emph{slower} on the logarithm
and permutations benchmarks). The situation was improved drastically, however, when we switched to the tagless version.

Yet, in comparison with faster-miniKanren, our implementation is still lagging behind. We can conclude that the optimizations
used in the Scheme/Racket version, have a different impact in the OCaml case; we save this problem for future research.


\section{Conclusion}

We presented a strongly-typed implementation of \miniKanren for OCaml. Our implementation
passes all tests written for \miniKanren (including those for disequality constraints);
in addition we implemented many interesting relational programs known from
the literature. We claim that our implementation can be used both as a convenient
relational DSL for OCaml and an experimental framework for future research in the area of
relational programming.

%We also want to express our gratitude to William Byrd, who infected us with relational programming,
%and for the extra time he sacrificed as both our tutor and friend.


\nocite{*}
\bibliographystyle{eptcs}
\bibliography{pat-match}

\end{document}
