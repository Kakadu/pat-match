% !TEX TS-program = pdflatex
% !TeX spellcheck = en_US
% !TEX root = main.tex

\begin{figure}[t]
\centering
\begin{subfigure}[t]{0.4\linewidth}
\centering
\begin{lstlisting}[xleftmargin=.7cm,linewidth=3.5cm]
match x, y with
| T, _ -> 1
| _, _ -> 2
\end{lstlisting}
%\vskip18.5mm
\caption{Pattern matching}
\label{fig:match-example-small-functional}
\end{subfigure}
\hspace{.5cm}
\begin{subfigure}[t]{0.5\linewidth}
\centering
\begin{lstlisting}[xleftmargin=1.7cm,linewidth=3.5cm]
if x 
then 1 
else 2
\end{lstlisting}
%\vskip8.5mm
\caption{Optimal implementation}
%\label{fig:matching-example3}
\end{subfigure}
\caption{Simple patterns matching example} 
\label{fig:match-example-small}
\end{figure}

\begin{figure}[t]
\begin{subfigure}[t]{1.0\linewidth}
\centering
\begin{lstlisting}
let naive_rel q rez =
  conde
    [ fresh (tmp) 
        (q === pair !!true tmp) 
        (rez === !!1)
    ; fresh (tmp l r) 
        (q =/= pair !!true tmp) 
        (q === pair l r)
        (rez === !!2) 
    ]
\end{lstlisting}
\caption{Na\"{i}ve compilation with disequality constraints}
\label{fig:match-compilation-small-diseq}
\end{subfigure}
\begin{subfigure}[t]{1.0\linewidth}
\centering
\begin{lstlisting}
let better_rel q rez =
  conde
    [ fresh () 
        (q === pair !!true __) 
        (rez === !!1)
    ; fresh () 
        (q =/= pair !!true __) 
        (rez === !!2) 
    ]
\end{lstlisting}
\caption{Better compilation with wildcards}
\label{fig:match-compilation-small-wc}
\end{subfigure}
\caption{Pattern matching compilation example } 
\label{fig:match-compilation-small}
\end{figure}

\begin{figure}[t]
\begin{subfigure}[t]{0.4\linewidth}
\centering
\begin{lstlisting}
match x, y, z with
| _, F, T -> 1
| F, T, _ -> 2
| _, _, F -> 3
| _, _, T -> 4
\end{lstlisting}
\vskip4.5mm
\caption{Pattern matching}
\label{fig:matching-example1}
\end{subfigure}
% \hspace{0.2cm}
\begin{subfigure}[t]{0.5\linewidth}
\centering
\begin{lstlisting}
if y then
  if x then
    if z then 4 else 3
  else 2
else
  if z then 1 else 3
\end{lstlisting}
%\vskip13.5mm
\caption{Optimal implementation}
\label{fig:matching-example2}
\end{subfigure}
\caption{Pattern matching compilation example (from~\cite{maranget2001})} 
\label{fig:match-example}
\end{figure}

\begin{figure*}[t]
%\centering
\begin{subfigure}{0.49\textwidth}
\begin{lstlisting}
let enc_with_diseq q rez =
  let _T = !!true in
  let _F = !!false in
  let w = Std.triple in 
  conde
    [ fresh (fresh1) (rez === !!1) 
        (q === w fresh1 _F _T)
    ; fresh (fresh1 x) (rez === !!2) 
        (q === w _F _T fresh1) 
        (q =/= w x _F _T)
    ; fresh (fresh1 fresh2 x y z) (rez === !!3) 
        (q === w fresh1 fresh2 _F) 
        (q =/= w x _F _T) 
        (q =/= w _F _T z)
    ; fresh (x y z fresh1 fresh2) (rez === !!4)
        (q =/= w x _F _T)
        (q =/= w _F _T z)
        (q =/= w x y _F)
        (q === w fresh1 fresh2 _T) ]
\end{lstlisting}
\caption{With disequality constraints}
\label{fig:matching-example3}
\end{subfigure}
%\vskip13.5mm
\begin{subfigure}{0.49\textwidth}
\begin{lstlisting}
let enc_with_wc q rez =
    let _T = !!true in
    let _F = !!false in
    let w = Std.triple in
    conde
      [ fresh () (rez === !!1) 
          (q === w __ _F _T)
      ; fresh () (rez === !!2) 
          (q === w _F _T __) 
          (q =/= w __ _F _T)
      ; fresh () (rez === !!3) 
          (q === w __ __ _F) 
          (q =/= w __ _F _T) 
          (q =/= w _F _T __)
      ; fresh () (rez === !!4)
          (q =/= w __ _F _T)
          (q =/= w _F _T __)
          (q =/= w __ __ _F)
          (q === w __ __ _T) ]
\end{lstlisting}
\caption{With wildcards}
%\label{fig:matching-example3}
\end{subfigure}
\caption{Two possible encodings of an example from Fig.~\ref{fig:match-example}}
\label{fig:maranget-example-compilation}
%\vskip13.5mm
\end{figure*}

\begin{figure*}[t!]
%\centering
\begin{subfigure}{0.49\textwidth}
\begin{lstlisting}[escapechar=|,numbers=left]
q=((_.13, false, true), 1);
q=((false, true, _.15), 2);
q=((_.13, _.14, false), 3);  |\label{line:three}|
q=((_.13 [=/= false; =/= _.22], _.14, true), 4);
q=((_.13 [=/= false], _.14, false), 3);
q=((_.13 [=/= _.22], _.14, true), 4);
q=((_.13, _.14 [=/= true], false), 3);
q=((_.13 [=/= _.22], _.14 [=/= true], true), 4);        |\label{line:eight}|
q=((_.13 [=/= false], _.14 [=/= false], true), 4);
q=((_.13, _.14 [=/= false], true), 4);                  |\label{line:ten}|
q=((_.13, _.14 [=/= false; =/= true], true), 4);
\end{lstlisting}
\caption{With disequality constraints}
%\label{fig:matching-example3}
\end{subfigure}
%\vskip13.5mm
\begin{subfigure}{0.49\textwidth}
\begin{lstlisting}
q=((_.13, false, true), 1);
q=((false, true, _.15), 2);


q=((_.13 [=/= false], _.14, false), 3);

q=((_.13, _.14 [=/= true], false), 3);
q=((_.13 [=/= false], _.14 [=/= false], true), 4);


q=((_.13, _.14 [=/= false; =/= true], true), 4);
\end{lstlisting}
\caption{With wildcards}
%\label{fig:matching-example3}
\end{subfigure}
\caption{The output of running two encodings from Figure~\ref{fig:maranget-example-compilation} where scrutinee is a triple of three fresh variables \lstinline|(_.13, _.14, _.15)|. One can observe that disequality constraints generate more (bogus) answers. Also, the last answer raises the idea that finite domain constraints may be useful for that example.}
%\vskip13.5mm
\label{fig:matching-maranget-compilation}
\end{figure*}

\section{\noCanren{}}
\label{sec:noCanren}

Translation of pattern matching from functional program to relational \lstinline|conde| clauses could be non-trivial if a few branches overlap. 
For a functional program from Figure~\ref{fig:match-example-small-functional} one could try a straightforward encoding without any constraints:

\begin{lstlisting}
let straightforward q rez =
  conde
    [ fresh () 
        (rez === !!1)
        (q === Std.pair !!true __) 
    ; fresh (l r)  
        (rez === !!2)
        (q === Std.pair l r)
    ]
\end{lstlisting}

\noindent Unfortunately, this encoding to relational program will have a different semantics from functional one. If scrutinee is a pair of \lstinline|true|'s, the stream of answers will have both \lstinline|(rez===1)| and \lstinline|(rez===2)| from the first and the second branches of \lstinline|conde|, respectively. The first answer is expected, but the second one contradicts behaviour of the functional program: the second branch should not be tested if previous one fits the scrutinee.
This issue appears only if branches overlap, and to get a proper semantics the approach of relational conversion~\cite{RelConversion} and associated tool\footnote{\url{https://github.com/Lozov-Petr/noCanren} (accessed: August 1st, 2022)} decided to forbid overlapping branches and not to use any constraints in generated code. 
The developer needs to manually rewrite branches of pattern matching, which is annoying and could lead to exponential increasing of the number of branches.

For an example shown in Figure~\ref{fig:match-example-small} we also provide encodings (Figure~\ref{fig:match-compilation-small}) with default disequality constraints and with wildcards.

An execution of this pattern matching on four possible ground scrutinees, encoding using wildcards provides four expected results: if first element of pair is \lstinline{true}, return 1; on \lstinline{false} -- 2.
Running na\"ive compilation scheme (Figure~\ref{fig:match-compilation-small-diseq}) discovers two more answers: pairs \lstinline|(true,false)| and \lstinline|(true,true)| may return 2.
Indeed, two branches of conde are not ordered, and running this scrutinees will not fail a disequality constraint: it will be simplified to \lstinline|=/= true| and \lstinline|=/= false| respectively.
From this example one could conclude that disequality constraints are not expressive enough to handle relational conversion of pattern matching.

One also could try to encode a more complicated example from~\cite{maranget2001} with four branches and a scrutinee being a triple of booleans (Figure~\ref{fig:match-example}).
In Figure~\ref{fig:maranget-example-compilation} one could observe relational encoding of this matching, 
and in Figure~\ref{fig:matching-maranget-compilation} the result of querying where a scrutinee is a triple of three fresh variables.
One could see that all answers returned by ``wildcardful'' relation are also returned by a relation with disequality constraints.
Extra answers on the left should be considered bogus.
For example, answer in line~\ref{line:three} about branch 3 is incorrect: the described scrutinee should be already matched by branch 2.
Also, answers in lines~\ref{line:eight} and~\ref{line:ten} should subsume each other.
Finally, the last answer demonstrates the need of finite domain constraints which allows considering disequality constraint containing ambivalent information such as \lstinline{[=/= false; =/= true]} violated.

The above demonstrates how \emph{wildcard} usage allows one to compile pattern matching in a very elegant way without the need of disjoint \lstinline{conde}-s.