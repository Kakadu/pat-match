% !TEX TS-program = xelatex
% !TeX spellcheck = en_US
% !TEX root = wildcards.tex
\documentclass[aspectratio=169
  , xcolor={svgnames}
  , hyperref=
      { colorlinks
      , urlcolor=DarkBlue
      }
  , russian  % This line affects translation of theorem titles
  ]{beamer}
\usepackage[svgnames]{xcolor}
\usetheme{CambridgeUS}
\usefonttheme{professionalfonts}


\makeatletter
\@ifclassloaded{beamer}{
  % get rid of header navigation bar
  \setbeamertemplate{headline}{}
  % get rid of bottom navigation symbols
  \setbeamertemplate{navigation symbols}{}
  % get rid of footer
  %\setbeamertemplate{footline}{}
  \setbeamertemplate{section in toc} {\inserttocsectionnumber.~\inserttocsection}

}
{}
\makeatother
%%%%%%%%%%%%%%%%%%%%%%%%%%%%%%%%%%%%%%%%%%%%%
\usepackage{fontawesome}
% \newfontfamily{\FA}{Font Awesome 5 Free} % some glyphs missing
\expandafter\def\csname faicon@facebook\endcsname{{\FA\symbol{"F09A}}}
\def\faQuestionSign{{\FA\symbol{"F059}}}
\def\faQuestion{{\FA\symbol{"F128}}}
\def\faExclamation{{\FA\symbol{"F12A}}}
\def\faUploadAlt{{\FA\symbol{"F093}}}
\def\faLemon{{\FA\symbol{"F094}}}
\def\faPhone{{\FA\symbol{"F095}}}
\def\faCheckEmpty{{\FA\symbol{"F096}}}
\def\faBookmarkEmpty{{\FA\symbol{"F097}}}

\newcommand{\faGood}{\textcolor{ForestGreen}{\faThumbsUp}}
\newcommand{\faBad}{\textcolor{red}{\faThumbsODown}}
\newcommand{\faWrong}{\textcolor{red}{\faTimes}}
\newcommand{\faMaybe}{\textcolor{blue}{\faQuestion}}
\newcommand{\faCheckGreen}{\textcolor{ForestGreen}{\faCheck}}
%%%%%%%%%%%%%%%%%%%%%%%%%%%%%%%%%%%%%%%%%%%%%

\usepackage{fontspec}
\usepackage{xunicode}
\usepackage{xltxtra}
\usepackage{xecyr}
\usepackage{hyperref}

\setmainfont[
 Ligatures=TeX,
 Extension=.otf,
 BoldFont=cmunbx,
 ItalicFont=cmunti,
 BoldItalicFont=cmunbi,
% Scale = 1.1
]{cmunrm}
\setsansfont[
 Ligatures=TeX,
 Extension=.otf,
 BoldFont=cmunsx,
 ItalicFont=cmunsi,
%  Scale = 1.2
]{cmunss}
%\setmainfont[Mapping=tex-text]{DejaVu Serif}
%\setsansfont[Mapping=tex-text]{DejaVu Sans}
%\setmonofont{Fira Code}[Contextuals=AlternateOff]
\setmonofont{Fira Code}[Contextuals=Alternate,Scale=0.9]
\newfontfamily{\myfiracode}[Scale=1.5,Contextuals=Alternate]{Fira Code}
%\setmonofont[Scale=0.9,BoldFont={Inconsolata Bold}]{Inconsolata}

\usepackage{polyglossia}
\setmainlanguage{russian}
\setotherlanguage{english}


%\newfontfamily\dejaVuSansMono{DejaVu Sans Mono}
% https://github.com/vjpr/monaco-bold/raw/master/MonacoB/MonacoB.otf
%\newfontfamily\monacoB{MonacoB}
%%%%%%%%%%%%%%%%%%%%%%%%%%%%%%%%%%%%%%%%%%%%%%%5
\usepackage{soul} % for \st that strikes through
\usepackage[normalem]{ulem} % \sout

\usepackage{stmaryrd}
\newcommand{\sem}[1]{\ensuremath{\llbracket #1\rrbracket}}

% We should use the following commands as \ocaml{} to prevent chewing
% extra space
\newcommand{\ocaml}{\textsc{OCaml}}
\newcommand{\OCaml}{\ocaml}
\newcommand{\haskell}{\textsc{Haskell}}
\newcommand{\Haskell}{\haskell}
\newcommand{\Scheme}{\textsc{Scheme}}
\newcommand{\miniKanren}{\textsc{miniKanren}}
\newcommand{\OCanren}{\textsc{OCanren}}
\newcommand{\noCanren}{\textsc{noCanren}}

\usepackage{listings}
%\lstdefinestyle{style1}{
%  language=haskell,
%  numbers=left,
%  stepnumber=1,
%  numbersep=10pt,
%  tabsize=4,
%  showspaces=false,
%  showstringspaces=false
%}
%\lstdefinestyle{hsstyle1}
%{ language=haskell
%%          , basicstyle=\monacoB
%         , deletekeywords={Int,Float,String,List,Void}
%         , breaklines=true
%         , columns=fullflexible
%         , commentstyle=\color{ForestGreen}
%         , escapeinside=§§
%         , escapebegin=\begin{russian}\commentfont
%         , escapeend=\end{russian}
%         , commentstyle=\color{ForestGreen}
%         , escapeinside=§§
%         , escapebegin=\begin{russian}\color{ForestGreen}
%         , escapeend=\end{russian}
%         , mathescape=true
%%          , backgroundcolor = \color{MyBackground}
%}
%
%\newcommand{\inline}[1]{\lstinline{haskell}{#1}}
%\def\hsinline{\mintinline{haskell}}
%\def\inline{\hsinline}
%
%\lstnewenvironment{hslisting} {
%    \lstset { style={hsstyle1} }
%  }
%  {}
%
%%%%%%%%%%%%%%%%%%%%%%%%%%%%%%%%%%%%%%%%%%%%%%%%%%%%%%%%%%%
%%\setmainfont[
%% Ligatures=TeX,
%% Extension=.otf,
%% BoldFont=cmunbx,
%% ItalicFont=cmunti,
%% BoldItalicFont=cmunbi,
%%]{cmunrm}
%%% С засечками (для заголовков)
%%\setsansfont[
%% Ligatures=TeX,
%% Extension=.otf,
%% BoldFont=cmunsx,
%% ItalicFont=cmunsi,
%%]{cmunss}
%% \setmonofont[Scale=0.6]{Monaco}
%
%\usefonttheme{professionalfonts}
%\usepackage{times}
\usepackage{tikz}
\usetikzlibrary{cd}
\usepackage{tikz-cd}
\usepackage{caption}
\usepackage{subcaption}

%\renewtheorem{definition}{برهان}[chapter]
%%\DeclareMathOperator{->}{\rightarrow}
%\newcommand\iso{\ensuremath{\cong}}
%\usepackage{verbatim}
%\usepackage{graphicx}
%\usetikzlibrary{arrows,shapes}

%\usepackage{amsmath}
%\usepackage{amsfonts}
\usepackage{scalerel}
\DeclareMathOperator*{\myvee}{\scalerel*{\vee}{\sum}}
\DeclareMathOperator*{\mywedge}{\scalerel*{\wedge}{\sum}}

%
%\usepackage{tabulary}
%
%% sudo aptget install ttf-mscorefonts-installer
%%\setmainfont{Times New Roman}
%%\setsansfont[Mapping=tex-text]{DejaVu Sans}
%
%%\setmonofont[Scale=1.0,
%%    BoldFont=lmmonolt10-bold.otf,
%%    ItalicFont=lmmono10-italic.otf,
%%    BoldItalicFont=lmmonoproplt10-boldoblique.otf
%%]{lmmono9-regular.otf}
%
\usepackage[cache=true]{minted}
\usemintedstyle{perldoc}

\def\hsinline{\mintinline{haskell}}
\def\mlinline{\mintinline[escapeinside=||]{ocaml}}
% color options
\definecolor{YellowGreen} {HTML}{B5C28C}
\definecolor{ForestGreen} {HTML}{009B55}
\definecolor{MyBackground}{HTML}{F0EDAA}


\author{Косарев Дмитрий}
\institute{матмех СПбГУ}

%\addtobeamertemplate{title page}{}{
%  \begin{center}{\tiny Дата сборки: \today}\end{center}
%}

\let\thefootnote\relax\footnotetext{Put your text here}

%\newcommand{\inbr}[1]{\left<{#1}\right>}
%\def\transarrow{\xrightarrow}
%\newcommand{\setarrow}[1]{\def\transarrow{#1}}
%\newcommand{\trans}[3]{{#1}\transarrow{\padding{\textstyle #2}\padding}\subarrow{#3}}
\usepackage{amsmath, amsthm}

\def\transarrow{\xrightarrow}
\newcommand{\setarrow}[1]{\def\transarrow{#1}}

\def\padding{\phantom{X}}
\newcommand{\setpadding}[1]{\def\padding{#1}}

\def\subarrow{}
\newcommand{\setsubarrow}[1]{\def\subarrow{#1}}


\newcommand{\trule}[2]{\dfrac{#1}{#2}}
\newcommand{\crule}[3]{\dfrac{#1}{#2},\;{#3}}
\newcommand{\withenv}[2]{{#1}\vdash{#2}}
\newcommand{\trans}[3]{{#1}\transarrow{\padding{\textstyle #2}\padding}\subarrow{#3}}
\newcommand{\ctrans}[4]{{#1}\transarrow{\padding#2\padding}\subarrow{#3},\;{#4}}
\newcommand{\llang}[1]{\mbox{\lstinline[mathescape]|#1|}}
\newcommand{\pair}[2]{\inbr{{#1}\mid{#2}}}
\newcommand{\inbr}[1]{\left<{#1}\right>}
\newcommand{\highlight}[1]{\color{red}{#1}}
%\newcommand{\ruleno}[1]{\eqno[\scriptsize\textsc{#1}]}
\newcommand{\ruleno}[1]{\mbox{[\textsc{#1}]}}
\newcommand{\rulename}[1]{\textsc{#1}}
\newcommand{\inmath}[1]{\mbox{$#1$}}
\newcommand{\lfp}[1]{fix_{#1}}
\newcommand{\gfp}[1]{Fix_{#1}}
\newcommand{\vsep}{\vspace{-2mm}}
\newcommand{\supp}[1]{\scriptsize{#1}}
\newcommand{\sembr}[1]{\llbracket{#1}\rrbracket}
\newcommand{\cd}[1]{\texttt{#1}}
\newcommand{\free}[1]{\boxed{#1}}
\newcommand{\binds}{\;\mapsto\;}
\newcommand{\dbi}[1]{\mbox{\bf{#1}}}
\newcommand{\sv}[1]{\mbox{\textbf{#1}}}
\newcommand{\bnd}[2]{{#1}\mkern-9mu\binds\mkern-9mu{#2}}
\newcommand{\meta}[1]{{\mathcal{#1}}}
\newcommand{\dom}[1]{\mathtt{dom}\;{#1}}
%\newcommand{\primi}[2]{\mathbf{#1}\;{#2}}
\renewcommand{\dom}[1]{\mathcal{D}om\,({#1})}
\newcommand{\ran}[1]{\mathcal{VR}an\,({#1})}
\newcommand{\fv}[1]{\mathcal{FV}\,({#1})}
\newcommand{\tr}[1]{\mathcal{T}r_{#1}}
\newcommand{\diseq}{\not\equiv}
\newcommand{\reprfunset}{\mathcal{R}}
\newcommand{\reprfun}{\mathfrak{f}}
\newcommand{\cstore}{\Omega}
\newcommand{\cstoreinit}{\cstore_\epsilon^{init}}
\newcommand{\csadd}[3]{add(#1, #2 \diseq #3)}  %{#1 + [#2 \diseq #3]}
\newcommand{\csupdate}[2]{update(#1, #2)}  %{#1 \cdot #2}
\newcommand{\primi}[1]{\mathbf{#1}}
\renewcommand{\sem}[1]{\llbracket #1 \rrbracket}
\newcommand{\ir}{\ensuremath{\mathcal{S}}}
\usepackage{tikz}
\newcommand*\circled[1]{\tikz[baseline=(char.base)]{
   \node[shape=circle,draw,inner sep=1pt] (char) {#1};}}

\let\emptyset\varnothing
\let\eps\varepsilon

%%%%%%%%%%%%%%%%%%%%%%%%%%%%%%%%%%%%%%%%

\title{Wildcards Logic Variables}

\date{  Thu 15 Sep 2022}

\author{\textbf{Dmitrii Kosarev}$^1$, Daniil Berezun$^2$, Peter Lozov$^1$}
\institute{St. Petersbur State University, Russia$^1$ \and JetBrains Research, The Netherlands$^2$}


\AtBeginSection[]
{
  \begin{frame}<beamer>
    \frametitle{Оглавление}
    \tableofcontents[currentsection,currentsubsection]
  \end{frame}
}

\newcommand{\verbatimfont}[1]{\def\verbatim@font{#1}}
\usepackage{verbatimbox}

%\setbeamertemplate{section in toc}{\inserttocsectionnumber.~\inserttocsection}
\begin{document}
\maketitle

% For every picture that defines or uses external nodes, you'll have to
% apply the 'remember picture' style. To avoid some typing, we'll apply
% the style to all pictures.
%\tikzstyle{every picture}+=[remember picture]

% By default all math in TikZ nodes are set in inline mode. Change this to
% displaystyle so that we don't get small fractions.
\everymath{\displaystyle}

% Uncomment these lines for an automatically generated outline.
\begin{frame}{В этих слайдах}


\begin{itemize}
  \item Interpeters, deseq in Scheme
  \item diseq in OCanren
  \item How wildcards work
  \item Peano example
  \item noCanren example
  \item mathcing example

\end{itemize}


\end{frame}

\begin{frame}[fragile]{Положение ФП в мире}

\begin{minted}{scheme}
(define eval-exp-o
 (lambda (exp env val)
  (conde
     ...
     (== `(quote ,v) exp)
     ...
     (== `(list . ,a*) exp)
     ...
     (== `(,rator ,rand) exp)
     ...
     (== `(lambda (,x ) ,body) exp)
     ...
\end{minted}


\footnote{William E. Byrd, Eric Holk, and Daniel P. Friedman.
  miniKanren, Live and Untagged: Quine Generation via Relational Interpreters (Programming Pearl). 2012}
\end{frame}



\begin{frame}[fragile]{23}

1


\end{frame}

\begin{frame}{Хотя на самом деле нужно рисовать вот так}
%\begin{center}
%\begin{tikzpicture}
% \node (n1)    at ( 0, 0) {C/C++};
% \node (n2)    at ( 6.5,-2.5) {C\#, Java, ...};
% \node (n3)    at ( 4,-3.5) {отстой};
% \node (n2)    at ( 10.34, 0) { OCaml, Haskell, и т.д. };
%  \node     at ( 8, -1) { Rust ? };
% \node [align=left]    at (10,1.5) { \parbox[l]{4cm}{Языки с зависимыми типами: Coq, Agda, и т.п. }};
% \draw (.2,-.2) .. controls (4,-4) .. (9, .8);
% \draw [stealth-stealth,line width=1.1pt](-3,-4) -- (10,-4);
%       \node   at (8.2,-4.5) {более high-level};
%       \node   at (-2,-4.5) {менее high-level};
%\end{tikzpicture}
%\end{center}
\end{frame}

\section{Синтаксис вызова функций}






%%%%%%%%%%%%%%%%%%%%%%%%%%%%%%%%%%%%%%%%%%%%

\section{Алгебраические типы данных}

\begin{frame}[fragile]{Алгебраические типы данных}
Алгебраические типы данных -- это скрещивание enum'ов и структур из Си.
\begin{block}{Пример: список значений типа \mlinline{'a}}
Связный список -- это структура данных, которая представляет собой \emph{\underline{или}}
пустой список (nil), \emph{\underline{или}} элемент списка (cons), который \emph{\underline{состоит из}} элемента типа \mlinline{'a}, хранящегося в списке, \emph{\underline{и}} ещё одного списка (хвоста).
\end{block}
\vspace{1em}

\begin{minipage}{0.45\linewidth}
\begin{minted}{ocaml}
type 'a list  =
  | Nil
  | Cons of 'a * 'a list
\end{minted}
(Полиморфный) тип \mlinline{'a list} списков значений типа \mlinline{'a}
\end{minipage}\hspace{1cm}
\begin{minipage}{0.45\linewidth}
\mlinline{Nil} и  \mlinline{Cons} -- \emph{конструкторы} типа \mlinline{'a list}\\

У конструктора  \mlinline{Nil} нет аргументов.\\

У конструктора  \mlinline{Cons} два  аргумента с типами \mlinline{'a} и  \mlinline{'a list}.\\
%\begin{minted}{ocaml}
%let make : a -> b -> c = fun a b -> ...
%let make2 : b -> c =
%  let a : a = ... in
%  make a
%\end{minted}
%Из функции \mlinline{make2} вернули частично примененную функцию, а аргумент хранится в \emph{замыкании (closure)}
\end{minipage}
\end{frame}


\begin{frame}[fragile]{Пример: (односвязные) списки с синтаксический сахар для них}
\begin{minipage}[t]{0.45\linewidth}
\begin{minted}{ocaml}
type 'a list =
  | Nil
  | Cons of 'a * 'a list
\end{minted}
\parbox[][40pt][t]{\linewidth}{
\vspace{1em}
Вот так будете писать большинство своих типов данных
}
\vspace{1em}
\begin{minted}{ocaml}
Nil : 'a list
Cons (1, Nil) : int list

Cons (2, Cons (1, Nil)) : int list

\end{minted}
\end{minipage}\hspace{1cm}
\begin{minipage}[t]{0.45\linewidth}
\begin{minted}{ocaml}
type 'a list =
  | []
  | (::) of 'a * 'a list
\end{minted}
\parbox[][40pt][t]{\linewidth}{
\vspace{1em}
Здесь специально выбранный конструктор, чтобы он был инфиксным и право ассоциативным
}
\vspace{1em}
\begin{minted}{ocaml}
[]       : 'a list
1::[]    : int list
[1]      : int list
1::2::[] : int list
[1;2]    : int list
\end{minted}
\end{minipage}
\end{frame}

\defverbatim[colored]{\demoOption}{
\begin{minted}{ocaml}
let run_on_two x y f =
  match x with
  | None -> ()
  | Some x ->
      match y with
      | None -> ()
      | Some y -> f x y

(* somewhere in .mli file *)
val run_on_two:
  'a option ->
  'b option ->
  ('a -> 'b -> unit) ->
  unit
\end{minted}
}

\defverbatim[colored]{\optionDefinition}{
\begin{minted}{ocaml}
type 'a option =
  | None
  | Some of 'a
\end{minted}
}


\begin{frame}{Пример: тип option}
\begin{minipage}[t]{0.45\linewidth}
\begin{minipage}[t]{0.45\linewidth}
\optionDefinition
\end{minipage}
\vspace{1em}

Либо нет значения (\mlinline=None=), либо какое-то есть (\mlinline=Some=)\\

Аналог nullptr, только его нельзя случайно разименовать\footnote{См. Tony Hoare ``Null References: The Billion Dollar Mistake``}
\end{minipage}\hspace{1cm}
\begin{minipage}[t]{0.45\linewidth}
\begin{minipage}[t]{0.45\linewidth}
\demoOption{}
\end{minipage}
\end{minipage}
\end{frame}


\defverbatim[colored]{\demoBoolean}{
\begin{minted}{ocaml}
(* Встроенный в OCaml *)
true : bool

(* Но можно описать свой *)
type boolean = True | False

(* И потом использовать *)
match ... with
| true ->
| false ->

if (* condition: bool *)
then ...
else ...
\end{minted}
}

\defverbatim[colored]{\booleanBlindness}{
\begin{minted}{ocaml}
val is_admin : user -> bool
val is_red : node -> true
val list_filter: ('a -> bool) ->
                 'a list -> 'a list

(*  Лучше как ниже *)
type role = Admin | User
val get_role : user -> role
val color = Red | Black
val get_color : node -> color

type filter = Keep | Remove
val list_filter: ('a -> filter) ->
                 'a list -> 'a list
\end{minted}
}

\begin{frame}{Пример: тип bool и ``Boolean blindness``}
\vspace{-1em}
\begin{minipage}[t]{0.45\linewidth}
\begin{minipage}[t]{0.35\linewidth}
\demoBoolean
\end{minipage}
\end{minipage}\hspace{.1cm}
\begin{minipage}[t]{0.35\linewidth}
\begin{minipage}[t]{0.5\linewidth}
\booleanBlindness{}
\end{minipage}
\end{minipage}
\end{frame}

\begin{frame}
\begin{center}
\Large{Почему алгебраические типы данных (язык Hope, 1980) называются "алгебраическими"?}
\vspace{5em}

%\tiny{(Вспоминаем понятие алгебраической структуры...)}
\end{center}
\end{frame}


\begin{frame}{Итоги}
Новые понятия
\begin{enumerate}
\item каррированные функции
\item вложенные функции
\item сопоставление с образцом и wildcard
\item хвостовая рекурсия
\item автоматический вывод типов
\item разделение на файлы интерфейса и реализации
\end{enumerate}
\end{frame}

\begin{frame}[allowframebreaks]
\frametitle<presentation>{Ссылки}
\begin{thebibliography}{10}
  \bibitem{blow}
    Предотвращая коллапс цивилизации (Preventing the Collapse of Civilization), 2019 (in English)
    \newblock {\em Jonathan Blow}
    \newblock\href{https://youtu.be/pW-SOdj4Kkk}{YouTube}
  \bibitem{cheatsheets}
    Шпаргалки по синтаксису OCaml
    \newblock {\em OCamlPro}
    \newblock\href{https://ocaml.org/docs/cheat_sheets.html}{4 PDF online}
  \bibitem{rwo}
    Книга Real World OCaml издания 2.0 (есть русский перевод издания 1.0)
    \newblock\href{https://dev.realworldocaml.org/toc.html}{online}
  \bibitem{javaTuringComplete}
    Java Generics are Turing Complete
    \newblock {\em Radu Grigore}
    \newblock\href{https://arxiv.org/pdf/1605.05274}{online}
  \bibitem{algDThistory}
    A Brief History of Algebraic Data Types
    \newblock {\em Li-yao Xia}
    \newblock\href{https://docs.google.com/presentation/d/131_CYsd9mEL-0XqXMyV7JeahWgtNCUV_JwjH7WqYFcI/edit}{online}
\end{thebibliography}
\end{frame}


\end{document}
