% !TEX TS-program = xelatex
% !TeX spellcheck = en_US
% !TEX root = wildcards.tex
\documentclass[aspectratio=169
  , xcolor={svgnames}
  , hyperref=
      { colorlinks
      , urlcolor=DarkBlue
      }
  , russian  % This line affects translation of theorem titles
  ]{beamer}
\usepackage[svgnames]{xcolor}
\usetheme{CambridgeUS}
\usefonttheme{professionalfonts}


\makeatletter
\@ifclassloaded{beamer}{
  % get rid of header navigation bar
  \setbeamertemplate{headline}{}
  % get rid of bottom navigation symbols
  \setbeamertemplate{navigation symbols}{}
  % get rid of footer
  %\setbeamertemplate{footline}{}
  \setbeamertemplate{section in toc} {\inserttocsectionnumber.~\inserttocsection}

}
{}
\makeatother
%%%%%%%%%%%%%%%%%%%%%%%%%%%%%%%%%%%%%%%%%%%%%
\usepackage{fontawesome}
% \newfontfamily{\FA}{Font Awesome 5 Free} % some glyphs missing
\expandafter\def\csname faicon@facebook\endcsname{{\FA\symbol{"F09A}}}
\def\faQuestionSign{{\FA\symbol{"F059}}}
\def\faQuestion{{\FA\symbol{"F128}}}
\def\faExclamation{{\FA\symbol{"F12A}}}
\def\faUploadAlt{{\FA\symbol{"F093}}}
\def\faLemon{{\FA\symbol{"F094}}}
\def\faPhone{{\FA\symbol{"F095}}}
\def\faCheckEmpty{{\FA\symbol{"F096}}}
\def\faBookmarkEmpty{{\FA\symbol{"F097}}}

\newcommand{\faGood}{\textcolor{ForestGreen}{\faThumbsUp}}
\newcommand{\faBad}{\textcolor{red}{\faThumbsODown}}
\newcommand{\faWrong}{\textcolor{red}{\faTimes}}
\newcommand{\faMaybe}{\textcolor{blue}{\faQuestion}}
\newcommand{\faCheckGreen}{\textcolor{ForestGreen}{\faCheck}}
%%%%%%%%%%%%%%%%%%%%%%%%%%%%%%%%%%%%%%%%%%%%%

\usepackage{fontspec}
\usepackage{xunicode}
\usepackage{xltxtra}
\usepackage{xecyr}
\usepackage{hyperref}

\setmainfont[
 Ligatures=TeX,
 Extension=.otf,
 BoldFont=cmunbx,
 ItalicFont=cmunti,
 BoldItalicFont=cmunbi,
% Scale = 1.1
]{cmunrm}
\setsansfont[
 Ligatures=TeX,
 Extension=.otf,
 BoldFont=cmunsx,
 ItalicFont=cmunsi,
%  Scale = 1.2
]{cmunss}
%\setmainfont[Mapping=tex-text]{DejaVu Serif}
%\setsansfont[Mapping=tex-text]{DejaVu Sans}
%\setmonofont{Fira Code}[Contextuals=AlternateOff]
\setmonofont{Fira Code}[Contextuals=Alternate,Scale=0.9]
\newfontfamily{\myfiracode}[Scale=1.5,Contextuals=Alternate]{Fira Code}
%\setmonofont[Scale=0.9,BoldFont={Inconsolata Bold}]{Inconsolata}

\usepackage{polyglossia}
\setmainlanguage{russian}
\setotherlanguage{english}


%\newfontfamily\dejaVuSansMono{DejaVu Sans Mono}
% https://github.com/vjpr/monaco-bold/raw/master/MonacoB/MonacoB.otf
%\newfontfamily\monacoB{MonacoB}
%%%%%%%%%%%%%%%%%%%%%%%%%%%%%%%%%%%%%%%%%%%%%%%5
\usepackage{soul} % for \st that strikes through
\usepackage[normalem]{ulem} % \sout

\usepackage{stmaryrd}
\newcommand{\sem}[1]{\ensuremath{\llbracket #1\rrbracket}}

% We should use the following commands as \ocaml{} to prevent chewing
% extra space
\newcommand{\ocaml}{\textsc{OCaml}}
\newcommand{\OCaml}{\ocaml}
\newcommand{\haskell}{\textsc{Haskell}}
\newcommand{\Haskell}{\haskell}
\newcommand{\Scheme}{\textsc{Scheme}}
\newcommand{\miniKanren}{\textsc{miniKanren}}
\newcommand{\OCanren}{\textsc{OCanren}}
\newcommand{\noCanren}{\textsc{noCanren}}

\usepackage{listings}
%\lstdefinestyle{style1}{
%  language=haskell,
%  numbers=left,
%  stepnumber=1,
%  numbersep=10pt,
%  tabsize=4,
%  showspaces=false,
%  showstringspaces=false
%}
%\lstdefinestyle{hsstyle1}
%{ language=haskell
%%          , basicstyle=\monacoB
%         , deletekeywords={Int,Float,String,List,Void}
%         , breaklines=true
%         , columns=fullflexible
%         , commentstyle=\color{ForestGreen}
%         , escapeinside=§§
%         , escapebegin=\begin{russian}\commentfont
%         , escapeend=\end{russian}
%         , commentstyle=\color{ForestGreen}
%         , escapeinside=§§
%         , escapebegin=\begin{russian}\color{ForestGreen}
%         , escapeend=\end{russian}
%         , mathescape=true
%%          , backgroundcolor = \color{MyBackground}
%}
%
%\newcommand{\inline}[1]{\lstinline{haskell}{#1}}
%\def\hsinline{\mintinline{haskell}}
%\def\inline{\hsinline}
%
%\lstnewenvironment{hslisting} {
%    \lstset { style={hsstyle1} }
%  }
%  {}
%
%%%%%%%%%%%%%%%%%%%%%%%%%%%%%%%%%%%%%%%%%%%%%%%%%%%%%%%%%%%
%%\setmainfont[
%% Ligatures=TeX,
%% Extension=.otf,
%% BoldFont=cmunbx,
%% ItalicFont=cmunti,
%% BoldItalicFont=cmunbi,
%%]{cmunrm}
%%% С засечками (для заголовков)
%%\setsansfont[
%% Ligatures=TeX,
%% Extension=.otf,
%% BoldFont=cmunsx,
%% ItalicFont=cmunsi,
%%]{cmunss}
%% \setmonofont[Scale=0.6]{Monaco}
%
%\usefonttheme{professionalfonts}
%\usepackage{times}
\usepackage{tikz}
\usetikzlibrary{cd}
\usepackage{tikz-cd}
\usepackage{caption}
\usepackage{subcaption}

%\renewtheorem{definition}{برهان}[chapter]
%%\DeclareMathOperator{->}{\rightarrow}
%\newcommand\iso{\ensuremath{\cong}}
%\usepackage{verbatim}
%\usepackage{graphicx}
%\usetikzlibrary{arrows,shapes}

%\usepackage{amsmath}
%\usepackage{amsfonts}
\usepackage{scalerel}
\DeclareMathOperator*{\myvee}{\scalerel*{\vee}{\sum}}
\DeclareMathOperator*{\mywedge}{\scalerel*{\wedge}{\sum}}

%
%\usepackage{tabulary}
%
%% sudo aptget install ttf-mscorefonts-installer
%%\setmainfont{Times New Roman}
%%\setsansfont[Mapping=tex-text]{DejaVu Sans}
%
%%\setmonofont[Scale=1.0,
%%    BoldFont=lmmonolt10-bold.otf,
%%    ItalicFont=lmmono10-italic.otf,
%%    BoldItalicFont=lmmonoproplt10-boldoblique.otf
%%]{lmmono9-regular.otf}
%
\usepackage[cache=true]{minted}
\usemintedstyle{perldoc}

\def\hsinline{\mintinline{haskell}}
\def\mlinline{\mintinline[escapeinside=||]{ocaml}}
% color options
\definecolor{YellowGreen} {HTML}{B5C28C}
\definecolor{ForestGreen} {HTML}{009B55}
\definecolor{MyBackground}{HTML}{F0EDAA}


\author{Косарев Дмитрий}
\institute{матмех СПбГУ}

%\addtobeamertemplate{title page}{}{
%  \begin{center}{\tiny Дата сборки: \today}\end{center}
%}

\let\thefootnote\relax\footnotetext{Put your text here}
%%%%%%%%%%%%%%%%%%%%%%%%%%%%%%%%%%%%%%%%

\title{Wildcards Logic Variables}
\subtitle{How to say in miniKanren that natural number is less than 5?}

\date{  Thu 15 Sep 2022}

\author
  [\textbf{Dmitrii Kosarev}, Daniil Berezun, Peter Lozov]
  {\textbf{Dmitrii Kosarev}$^1$, Daniil Berezun$^2$, Peter Lozov$^1$}
\institute[]{St. Petersbur State University, Russia$^1$ \and JetBrains Research, The Netherlands$^2$}


\AtBeginSection[]
{
  \begin{frame}<beamer>
    \frametitle{Оглавление}
    \tableofcontents[currentsection,currentsubsection]
  \end{frame}
}

\newcommand{\verbatimfont}[1]{\def\verbatim@font{#1}}
\usepackage{verbatimbox}

%\setbeamertemplate{section in toc}{\inserttocsectionnumber.~\inserttocsection}
\begin{document}
\maketitle

% For every picture that defines or uses external nodes, you'll have to
% apply the 'remember picture' style. To avoid some typing, we'll apply
% the style to all pictures.
%\tikzstyle{every picture}+=[remember picture]

% By default all math in TikZ nodes are set in inline mode. Change this to
% displaystyle so that we don't get small fractions.
\everymath{\displaystyle}

%\begin{frame}{В этих слайдах}
%
%
%\begin{itemize}
%  \item Interpeters, deseq in Scheme
%  \item diseq in OCanren
%  \item How wildcards work
%  \item Peano example
%  \item noCanren example
%  \item mathcing example
%\end{itemize}
%\end{frame}

\defverbatim[colored]{\schemeInterpreter}{
\begin{minted}{scheme}
; Scheme
(define eval-exp-o
 (lambda (exp env val)
  (conde
     ...
     (== `(quote ,v) exp)
     ...
     (== `(list . ,a*) exp)
     ...
     (== `(,rator ,rand) exp)
     ...
     (== `(lambda (,x ) ,body) exp)
     ...
\end{minted}
}

\defverbatim[colored]{\hackOne}{
\begin{minted}{scheme}
fresh (v)
  (=/= q `(quote ,v))
\end{minted}
}

\begin{frame}{A reminder of relational interpreters...}
\framesubtitle{}
\begin{figure}[t]
\begin{subfigure}{.49\textwidth}
\begin{minipage}{.4\textwidth}
\schemeInterpreter
\end{minipage}
\end{subfigure}
\hskip3em
\begin{subfigure}[ht]{.4\textwidth}
Question: how to say that expression \mintinline{scheme}|q| doesn't fit the first case (is not a quote expression)?

\begin{minipage}{.4\textwidth}
\hackOne
\end{minipage}
\vskip1em
It's possible because data representation specifies
\begin{itemize}
\item arity first
\item content of a list later
\end{itemize}
\end{subfigure}
\end{figure}
\footnote{W.E. Byrd et al. \miniKanren{}, Live and Untagged: Quine Generation via Relational Interpreters. 2012}
\end{frame}

\defverbatim[colored]{\XXX}{
\begin{minted}{ocaml}
(* very much simplified *)
type exp =
  | Quote of exp
  | List of exp list
  | Appl of exp * exp
  | Lam of var * exp * exp
  | ...
\end{minted}
}
\defverbatim[colored]{\BuggyIneq}{
\begin{minted}{ocaml}
(* not quite a solution *)
fresh (v) (q =/= quote v)
\end{minted}
}
\begin{frame}{\OCanren{} Uses Different Representation of Values}
\begin{figure}[ht]
\begin{subfigure}[t]{.45\textwidth}
\begin{minipage}{.4\textwidth}
\XXX
\end{minipage}
\vskip1em
In \OCanren{} we specify constructor \emph{first} and arity and arguments \emph{later}
\end{subfigure}\hskip2em
\begin{subfigure}[ht]{.49\textwidth}
  \textbf{Question}: how to say that expression \mintinline{ocaml}|q| doesn't fit the first case (is not a quote expression)?
  \begin{center}
  \begin{minipage}{.4\textwidth}
        \BuggyIneq
        \end{minipage}
  \end{center}

        We said: \emph{exists v such that \mintinline{ocaml}{q} is not \mintinline{ocaml}{quote v}}\\

        We wanted to say: \emph{\mintinline{ocaml}{q} is not \mintinline{ocaml}{quote v} for any possible \mintinline{ocaml}{v}.}\\

        In other words: $\forall$ v: \mlinline{q =/= quote v}.
\end{subfigure}
\end{figure}
\footnote{D. Kosarev and D. Boulytchev. Typed Embedding of a Relational  Language in \OCaml{}. 2016}
\end{frame}

\begin{frame}{Isn't this issue too ad hoc?}
\begin{itemize}
\item It may look like too specific for relational interpreters...
\item ... or to \OCanren{} itself
\item But we managed to found an example, that make sense bot for \Scheme{}/\miniKanren{} and \OCaml{}/\OCanren{}
\end{itemize}

\end{frame}

\begin{frame}{Example with Peano numbers}

What would be solutions for \mintinline{scheme}{q}, that is not equal \mintinline{scheme}{(s (s (s ,v))))} for any possible \mintinline{scheme}{v}?

\begin{itemize}
\item[\faGood] \mintinline{scheme}{z}
\item[\faGood] \mintinline{scheme}{(s z)}
\item[\faGood] \mintinline{scheme}{(s (s z))}
\item[\faBad] \mintinline{scheme}{(s (s (s z)))}
\item[\faBad] \mintinline{scheme}{(s (s (s (s z))))}
\item[\faBad] \mintinline{scheme}{(s (s (s (s (s z)))))}
\item ...
\item All peano numbers less than three fit, others doesn't.
\end{itemize}

\end{frame}
\begin{frame}{Contributions}
\begin{enumerate}


\item Wildcard Logic Variables
\begin{itemize}
\item Are able to express ``Peano number that is \emph{less} than constant N''
\item Previously, we could only do ``Peano number that is  \emph{greater} than constant N''
\end{itemize}
\item An application to \noCanren{} -- more expressivity
\item Work in progress application to pattern-matching synthesis
\end{enumerate}
\end{frame}

\begin{frame}{Implementation Sketch}
We introduce a special \emph{wildcard variables} (denoted by two underscores: \mlinline{__}) and

\begin{itemize}
\item \emph{modify unification}: wildcards unify with everything but don't extend current substitution
\item disequality constraints doesn't change significantly
\end{itemize}
%\begin{figure}[ht]
%\begin{subfigure}[t]{.45\textwidth}
%
%\end{subfigure}\hskip1em
%\begin{subfigure}[t]{.45\textwidth}
\vskip1em
Examples (\Scheme{}-like syntax):
\begin{itemize}
\item  \mlinline{(=/= (1 _.10)  (__ 2))} $\rightsquigarrow$
  \mlinline{(=/= _.10 2)}
\item \mlinline{(=/= (__  __) _.10)} \mlinline{(=== (_.11 _.12)  _.10)}
$\rightsquigarrow$ fail
\item A fresh variable \mlinline{(=/= _.10  __)}. \\
If a variable has non-empty domain, $\rightsquigarrow$ fail.\\
For empty domain $\rightsquigarrow$ success
\end{itemize}
\end{frame}


\begin{frame}{Related Works}
\begin{enumerate}
\item Eigen variables
\begin{itemize}
\item Universal quantifier inside unification, disequality constraints support is shaky\footnotemark
\end{itemize}
\item Universal Quantification and Implication\footnotemark
%\begin{itemize}
%\item Looks promising
%\end{itemize}
\end{enumerate}
\footnotetext[1]{W. E. Byrd. 2013. Relational Synthesis of Programs.}
\footnotetext[2]{E. Jin, G. Rosenblatt, M. Might, and Zhang L. 2021. Universal Quantification
and Implication in \miniKanren{}.}
%\footnote[frame,3]{Here is a footnote}
\end{frame}

\begin{frame}{An application: \noCanren}
\begin{itemize}
\item [\faBad] \miniKanren{} requires some time to study (as any language/DSL)
\item An idea: synthesize relational code from a subset of general purpose language (for example, \OCaml{})\footnote{P. Lozov, A. Vyatkin, and D. Boulytchev. 2018. Typed Relational Conversion.}
\begin{itemize}
\item[\faGood] Type checker for free
\item[\faGood] Convenient \OCaml{}-like syntax  for free (at least it should be)
\item Semantics preserving transformation to \miniKanren{} is proved on paper
\end{itemize}
\item  Restrictions on subset of \OCaml{} are quite strong
\begin{itemize}
\item Some language features are not available (pattern matching guards)
\item Some \OCaml{} best practices hurt performance
\item Overlapping patterns in pattern matching are banished\\
But wildcard variables will help with that \faGood

\end{itemize}
\end{itemize}
\end{frame}

\defverbatim[colored]{\lozovMLexOne}{
\begin{minted}{ocaml}
(* LozovML *)
match x,y with
| T, _ -> 1
| _, _ -> 2
\end{minted}
}
\defverbatim[colored]{\straightforwardCompilation}{
\begin{minted}{ocaml}
let straightforward q rez = (* OCanren syntax *)
  conde
    [ fresh ()
        (rez === !!1)
        (q === Std.pair !!true __)
    ; fresh (l r)
        (rez === !!2)
        (q === Std.pair l r)
    ]
\end{minted}
}
\defverbatim[colored]{\naiveCompilation}{
\begin{minted}{ocaml}
let naive_rel q rez = (* OCanren syntax *)
  conde
    [ fresh (tmp)
        (q === pair !!true tmp)
        (rez === !!1)
    ; fresh (tmp l r)
        (q =/= pair !!true tmp)
        (q === pair l r)
        (rez === !!2)
    ]
\end{minted}
}

\defverbatim[colored]{\betterCompilation}{
\begin{minted}{ocaml}
let better_rel q rez = (* OCanren syntax *)
  conde
    [ fresh ()
        (q === pair !!true __)
        (rez === !!1)
    ; fresh ()
        (q =/= pair !!true __)
        (rez === !!2)
    ]
\end{minted}
}

\begin{frame}{\noCanren{} example
  \only<1>{1}\only<2>{2}\only<3>{3}/3:
  \only<1>{no constraints at all}
  \only<2>{default disequality constrains}
  \only<3>{disequality with wildcard variables}
  }
\begin{figure}[ht]
\begin{subfigure}[t]{.25\textwidth}
  \begin{minipage}{1\textwidth}
      \vspace{-1em}
  Overlapping patterns are banned, let's see what could go wrong...
  \vspace{1em}
  \lozovMLexOne
  \end{minipage}
\vskip1em
\only<3>{

  An idea: for every branch
  \begin{itemize}
    \item put current pattern as is
    \item previous ones with \mlinline{=/=} and \mlinline{__}
  \end{itemize}
}
\end{subfigure}\hskip3em
\begin{subfigure}[t]{.65\textwidth}
\only<1>{
  \begin{minipage}{.55\textwidth}
  \straightforwardCompilation
  \end{minipage}
  \begin{itemize}
  \item [\faBad] no restrictions between matching branches, can ``return'' for \mlinline{(q===Std.pair !!true !!true)} both 1 and 2
  \end{itemize}
}
\only<2>{
  %\vskip{-0.1em}
  \begin{minipage}{.45\textwidth}
  \naiveCompilation
  \end{minipage}
  \begin{itemize}
  \item [\faBad] For \mlinline{fresh (v) (q === pair !!true v)} both 1 and 2, because variable \mlinline{tmp} is existential.

   Restrictions established by default disequality constraints are not powerful enough
  \end{itemize}
}
\only<3>{
  \begin{minipage}{.45\textwidth}
  \betterCompilation
  \end{minipage}
  \begin{itemize}
  \item [\faGood] If \mlinline{q} fits \mlinline{fresh (v) (q === pair !!true v)} the piece of code reaches \mlinline{rez === 1} and doesn't reach second branch (which is expected)
  \end{itemize}
}
\end{subfigure}
\end{figure}
\end{frame}

\begin{frame}{Application to Pattern Matching Synthesis}

\begin{figure}[ht]
\begin{subfigure}[t]{.49\textwidth}
Input language:
\[
 \begin{array}{rcll}
    \mathcal{C} & = & \{ C_1^{k_1}, \dots, C_n^{k_n} \}\\
    \mathcal{V} & = & \mathcal{C}\,\mathcal{V}^*\\
    \mathcal{P} & = & \_ \mid \mathcal{C}\,\mathcal{P}^*
 \end{array}
\]
Declarative semantics  $\trans{}{}{}$:
\[
\setarrow{\xrightarrow}
 \begin{array}{rcl}
    &\trans{\inbr{v;\,p_1,\dots,p_k}}{}{i} &   \\
    &1\leqslant i\leqslant k \\
    & v \in \mathcal{V}  \\
    & p_1,\dots,p_k \in \mathcal{P} &
 \end{array}
\]
\end{subfigure}
\begin{subfigure}[t]{.49\textwidth}
Language of compiled representation (semantics denoted as  $\setsubarrow{_{\mathcal S}}\trans{}{}{}$)
\[
\begin{array}{rcl}
  \mathcal M & =        & \bullet \\
             &    \mid  & \mathcal M\,[\mathbb{N}] \\
  \ir        & =        & \primi{return}\;\mathbb{N} \\
             &    \mid  & \primi{if}\;\mathcal{M}\;\primi{starts with}\;\mathcal{C}\;\primi{then}\; \ir\;\primi{else}\;\ir
\end{array}
\]
Synthesis task:
\begin{center}
$\setarrow{\xrightarrow}
\forall v\in \mathcal V,\; \forall 1\leqslant i\leqslant n: $\\
$\trans{\inbr{v;\,p_1,\dots,p_n}}{}{i}$\\
$\Longleftrightarrow$ \\
$\setsubarrow{_{\mathcal S}}\withenv{v}{\trans{\pi}{}{i}}$ \\
\end{center}

\end{subfigure}
\end{figure}
\end{frame}

\begin{frame}{$\forall$ Quantifier Elimination}
For a concrete pattern matching we know
\begin{itemize}
  \item type of a scrutinee
  \item depth of matching
\end{itemize}
\vskip1em
and we can replace $\forall$ by a finite conjunction of ground examples of scrutinee

\begin{itemize}
  \item [\faGood] Finite conjunction could finish...
  \item [\faBad] Exponential number of examples is really slow
\end{itemize}
\vskip2em
Long-term plan: replace a large finite conjunction by something smaller
\end{frame}


\begin{frame}[fragile]{An Attempt of encoding of Pattern Matching Synthesis with Wildcards }
\begin{figure}[ht]
  \begin{subfigure}[t]{.35\textwidth}
    \begin{minted}{ocaml}
match x, y, z with
| _, F, T -> 1
| F, T, _ -> 2
| _, _, F -> 3
| _, _, T -> 4
    \end{minted}

  \end{subfigure}
  \begin{subfigure}[t]{.49\textwidth}
    Branch 1
    \begin{minted}{ocaml}
     fresh (tmp1)
       (rez === 1)
       (scru === triple tmp1 F T)
   \end{minted}
    Branch 2
    \begin{minted}{ocaml}
    fresh (tmp1)
      (rez === 1)
      (scru === triple F T tmp1)
      (scru =/= triple __ F T)
    \end{minted}
    etc...
  \end{subfigure}
\end{figure}
\begin{itemize}
\item [\faGood] We managed in current branch to ``ban'' previous branches (by adding negative information)
\item [\faBad] Unifications (positive information) have existential holes which allows synthesizer to give undesired answers (in paper we call it ``conservative approximation of scrutinee'' )
\end{itemize}

\end{frame}

\begin{frame}{Technical remark 1}

\[
\begin{array}{rcl}
  \mathcal M & =        & \bullet \\
  &    \mid  & \mathcal M\,[\mathbb{N}] \\
  \ir        & =        & \primi{return}\;\mathbb{N} \\
  &    \mid  & \primi{if}\;\mathcal{M}\;\primi{starts with}\;\mathcal{C}\;\primi{then}\; \ir\;\primi{else}\;\ir
\end{array}
\]
While evaluating \textbf{if} expression above disequality constraints maybe give empty domain
\end{frame}

\begin{frame}{}
Some technical details about typing and wildcards in unification
\end{frame}



\begin{frame}{Results}

\begin{enumerate}
\item blah
\item blah
\item blah
\item blah

\end{enumerate}
\end{frame}

\begin{comment}


\begin{frame}[allowframebreaks]
\frametitle<presentation>{Ссылки}
\begin{thebibliography}{10}
  \bibitem{blow}
    Предотвращая коллапс цивилизации (Preventing the Collapse of Civilization), 2019 (in English)
    \newblock {\em Jonathan Blow}
    \newblock\href{https://youtu.be/pW-SOdj4Kkk}{YouTube}
  \bibitem{cheatsheets}
    Шпаргалки по синтаксису OCaml
    \newblock {\em OCamlPro}
    \newblock\href{https://ocaml.org/docs/cheat_sheets.html}{4 PDF online}
  \bibitem{rwo}
    Книга Real World OCaml издания 2.0 (есть русский перевод издания 1.0)
    \newblock\href{https://dev.realworldocaml.org/toc.html}{online}
  \bibitem{javaTuringComplete}
    Java Generics are Turing Complete
    \newblock {\em Radu Grigore}
    \newblock\href{https://arxiv.org/pdf/1605.05274}{online}
  \bibitem{algDThistory}
    A Brief History of Algebraic Data Types
    \newblock {\em Li-yao Xia}
    \newblock\href{https://docs.google.com/presentation/d/131_CYsd9mEL-0XqXMyV7JeahWgtNCUV_JwjH7WqYFcI/edit}{online}
\end{thebibliography}
\end{frame}
\end{comment}

\end{document}
