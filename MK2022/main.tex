% !TEX TS-program = pdflatex
% !TeX spellcheck = en_US
% !TEX root = main.tex
\documentclass
  [sigplan
  ,screen
  ,review
  ]{acmart}

\usepackage[
    type={CC},           % your choice
    modifier={by-sa},    % your choice
    version={4.0},       % your choice
]{doclicense}            % your choice, see \doclicenseThis below


\AtBeginDocument{%
  \providecommand\BibTeX{{%
    Bib\TeX}}}

%% Rights management information.  This information is sent to you
%% when you complete the rights form.  These commands have SAMPLE
%% values in them; it is your responsibility as an author to replace
%% the commands and values with those provided to you when you
%% complete the rights form.
\setcopyright{acmcopyright}
\copyrightyear{2022}
\acmYear{2022}
\acmDOI{XXXXXXX.XXXXXXX}

%% These commands are for a PROCEEDINGS abstract or paper.
\acmConference[miniKanren 2022]{The Fourth miniKanren and Relational Programming Workshop}{September 15, 2022}{Ljubljana, Slovenia}
\acmPrice{15.00}
\acmISBN{978-1-4503-XXXX-X/18/06}
\acmMonth{9}
\acmArticle{1} % your number in the order of presentations (between 1 and 11)

\usepackage{lineno}
%\linenumbers
% !TeX spellcheck = en_US
% !TEX root = main.tex

\usepackage{alltt}
\usepackage{latexsym}
\usepackage{array}
\usepackage{comment}
%\usepackage{makeidx}
%\usepackage{indentfirst}
%\usepackage{verbatim}
%\usepackage{color}
%\usepackage{url}
%\usepackage{xspace}
%\usepackage{hyperref}
%\usepackage{stmaryrd}
\usepackage{amsmath, amsthm}
%\usepackage{graphicx}
%\usepackage{euscript}
\usepackage{mathtools}
%\usepackage{mathrsfs}
%\usepackage{multirow,bigdelim}
%\usepackage{subcaption}
%\usepackage{placeins}
\usepackage{csvsimple}
\usepackage{array}

%% Some recommended packages.
\usepackage{booktabs}   %% For formal tables:
                       %% http://ctan.org/pkg/booktabs
\usepackage{subcaption} %% For complex figures with subfigures/subcaptions
                       %% http://ctan.org/pkg/subcaption
\usepackage{multirow}

\usepackage{placeins}

\usepackage{listings}
\lstdefinelanguage{ocanren}{
keywords={run, conde, fresh, let, in, match, with, when, class, type,
object, method, of, rec, repeat, until, while, not, do, done, as, val, inherit,
new, module, sig, deriving, datatype, struct, if, then, else, open, private, virtual, include, success, failure,
true, false},
sensitive=true,
commentstyle=\small\itshape\ttfamily,
keywordstyle=\ttfamily\textbf,
identifierstyle=\ttfamily,
basewidth={0.5em,0.5em},
columns=fixed,
mathescape=true,
fontadjust=true,
literate={fun}{{$\lambda$}}1 {->}{{$\to$}}3 {===}{{$\equiv$}}1 {=/=}{{$\not\equiv$}}1 {|>}{{$\triangleright$}}3 {\\/}{{$\vee$}}2 {/\\}{{$\wedge$}}2 {^}{{$\uparrow$}}1,
morecomment=[s]{(*}{*)}
}

\lstset{
%mathescape=true,
%basicstyle=\small,
%identifierstyle=\ttfamily,
%keywordstyle=\bfseries,
%commentstyle=\scriptsize\rmfamily,
%basewidth={0.5em,0.5em},
%fontadjust=true,
language=ocanren
}

\newcommand{\lstquot}[1]{``\lstinline{#1}''}
\newcommand{\sembr}[1]{\llbracket{#1}\rrbracket}
\newcommand\false{$f\!alse$}
\newcommand\myif{i\!f}


\def\transarrow{\xrightarrow}
\newcommand{\setarrow}[1]{\def\transarrow{#1}}

\def\padding{\phantom{X}}
\newcommand{\setpadding}[1]{\def\padding{#1}}

\def\subarrow{}
\newcommand{\setsubarrow}[1]{\def\subarrow{#1}}

\newcommand{\OCaml}{\textsc{OCaml}}
\newcommand{\OCanren}{\textsc{OCanren}}
\newcommand{\miniKanren}{\textsc{miniKanren}}
\newcommand{\Scheme}{\textsc{Scheme}}

\newcommand{\trule}[2]{\dfrac{#1}{#2}}
\newcommand{\crule}[3]{\dfrac{#1}{#2},\;{#3}}
\newcommand{\withenv}[2]{{#1}\vdash{#2}}
\newcommand{\trans}[3]{{#1}\transarrow{\padding{\textstyle #2}\padding}\subarrow{#3}}
\newcommand{\ctrans}[4]{{#1}\transarrow{\padding#2\padding}\subarrow{#3},\;{#4}}
\newcommand{\llang}[1]{\mbox{\lstinline[mathescape]|#1|}}
\newcommand{\pair}[2]{\inbr{{#1}\mid{#2}}}
\newcommand{\inbr}[1]{\left<{#1}\right>}
\newcommand{\highlight}[1]{\color{red}{#1}}
%\newcommand{\ruleno}[1]{\eqno[\scriptsize\textsc{#1}]}
\newcommand{\ruleno}[1]{\mbox{[\textsc{#1}]}}
\newcommand{\rulename}[1]{\textsc{#1}}
\newcommand{\inmath}[1]{\mbox{$#1$}}
\newcommand{\lfp}[1]{fix_{#1}}
\newcommand{\gfp}[1]{Fix_{#1}}
\newcommand{\vsep}{\vspace{-2mm}}
\newcommand{\supp}[1]{\scriptsize{#1}}
\renewcommand{\sembr}[1]{\llbracket{#1}\rrbracket}
\newcommand{\cd}[1]{\texttt{#1}}
\newcommand{\free}[1]{\boxed{#1}}
\newcommand{\binds}{\;\mapsto\;}
\newcommand{\dbi}[1]{\mbox{\bf{#1}}}
\newcommand{\sv}[1]{\mbox{\textbf{#1}}}
\newcommand{\bnd}[2]{{#1}\mkern-9mu\binds\mkern-9mu{#2}}
\newcommand{\meta}[1]{{\mathcal{#1}}}
\newcommand{\dom}[1]{\mathtt{dom}\;{#1}}
%\newcommand{\primi}[2]{\mathbf{#1}\;{#2}}
\renewcommand{\dom}[1]{\mathcal{D}om\,({#1})}
\newcommand{\ran}[1]{\mathcal{VR}an\,({#1})}
\newcommand{\fv}[1]{\mathcal{FV}\,({#1})}
\newcommand{\tr}[1]{\mathcal{T}r_{#1}}
\newcommand{\diseq}{\not\equiv}
\newcommand{\reprfunset}{\mathcal{R}}
\newcommand{\reprfun}{\mathfrak{f}}
\newcommand{\cstore}{\Omega}
\newcommand{\cstoreinit}{\cstore_\epsilon^{init}}
\newcommand{\csadd}[3]{add(#1, #2 \diseq #3)}  %{#1 + [#2 \diseq #3]}
\newcommand{\csupdate}[2]{update(#1, #2)}  %{#1 \cdot #2}
\newcommand{\primi}[1]{\mathbf{#1}}
\newcommand{\sem}[1]{\llbracket #1 \rrbracket}
\newcommand{\ir}{\ensuremath{\mathcal{S}}}
\usepackage{tikz}
\newcommand*\circled[1]{\tikz[baseline=(char.base)]{
   \node[shape=circle,draw,inner sep=1pt] (char) {#1};}}

\let\emptyset\varnothing
\let\eps\varepsilon

\newtheorem{Observation}{Observation}

\newcommand{\todo}[1]{\textcolor{blue}{#1}}
%\geometry{
%     top=18pt, bottom=14pt, inner=21pt, outer=21pt,
%     paperwidth=5.5in, paperheight=8.5in,
%     }

\settopmatter{printacmref=false,printfolios=false}
\fancyfoot{}

\makeatletter
\def\@formatdoi#1{}
\def\@permissionCodeOne{miniKanren.org/workshop}
\def\@copyrightpermission{\doclicenseThis} % your choice of text
\def\@copyrightowner{Copyright held by the author(s).} % your choice
\makeatother

\copyrightyear{2022}
\setcopyright{rightsretained}


\acmConference[miniKanren 2022]{The Fourth miniKanren and Relational Programming Workshop}{September 15 2022}{Online}

\acmMonth{8}
\acmArticle{8} % your number in the order of presentations (between 1 and 11)



%% Bibliography style
\bibliographystyle{ACM-Reference-Format}
%% Citation style
%% Note: author/year citations are required for papers published as an
%% issue of PACMPL.
\citestyle{acmauthoryear}   %% For author/year citations


%%%%%%%%%%%%%%%%%%%%%%%%%%%%%%%%%%%%%%%%%%%%%%%%%%%%%%%%%%%%%%%%%%%%%%
%% Note: Authors migrating a paper from PACMPL format to traditional
%% SIGPLAN proceedings format must update the '\documentclass' and
%% topmatter commands above; see 'acmart-sigplanproc-template.tex'.
%%%%%%%%%%%%%%%%%%%%%%%%%%%%%%%%%%%%%%%%%%%%%%%%%%%%%%%%%%%%%%%%%%%%%%

\sloppy


\begin{document}

\title[xxx]{Wildcard Variables}

%\titlenote{This work was partially supported by the grant 18-01-00380 from The Russian Foundation for Basic Research} %% \titlenote is optional;

%\author{John Doe}
%\affiliation{
%	\institution{Ice Crown University}
%	\country{Northrend}
%}
%\email{lichkind@wow.com}


%\author{Dmitry Kosarev}
%\affiliation{
%  \institution{Saint Petersburg State University}
%  \country{Russia}
%}
%\affiliation{
%  \institution{JetBrains Research}
%  \country{Russia}
%}
%\email{Dmitrii.Kosarev@pm.me}

%\author{Dmitry Boulytchev}
%\affiliation{
%	\institution{Saint Petersburg State University}
%	\country{Russia}
%}
%
%\affiliation{
%	\institution{JetBrains Research}
%	\country{Russia}
%}
%\email{dboulytchev@math.spbu.ru}

%% Abstract
%% Note: \begin{abstract}...\end{abstract} environment must come
%% before \maketitle command
\begin{abstract}
We propose a new kind of logic variables -- wildcard variables -- as a limited form of universal quantification. Combined with disequality constraints they extend the expressivity of \OCanren{} -- typed dialect of \miniKanren{}. We also report out progress on applying this idea to a task of synthesizing pattern matching compilation scheme.
\end{abstract}


%% 2012 ACM Computing Classification System (CSS) concepts
%% Generate at 'http://dl.acm.org/ccs/ccs.cfm'.
\begin{CCSXML}
<ccs2012>
<concept>
<concept_id>10011007.10011006.10011008.10011009.10011015</concept_id>
<concept_desc>Software and its engineering~Constraint and logic languages</concept_desc>
<concept_significance>500</concept_significance>
</concept>
<concept>
<concept_id>10011007.10011006.10011041.10011047</concept_id>
<concept_desc>Software and its engineering~Source code generation</concept_desc>
<concept_significance>500</concept_significance>
</concept>
</ccs2012>
\end{CCSXML}

\ccsdesc[500]{Software and its engineering~Constraint and logic languages}
\ccsdesc[500]{Software and its engineering~Source code generation}
%% End of generated code


%% Keywords
%% comma separated list
\keywords{relational programming, relational interpreters, pattern matching}  %% \keywords are mandatory in final camera-ready submission


%% \maketitle
%% Note: \maketitle command must come after title commands, author
%% commands, abstract environment, Computing Classification System
%% environment and commands, and keywords command.

\maketitle

\thispagestyle{empty}

% !TEX TS-program = pdflatex
% !TeX spellcheck = en_US
% !TEX root = main.tex
\section{Introduction}
\label{sec:intro}

Relational and logic programming are powerful techniques for enumerating the space of possible answers for a query. Constraints allow us to prune search space and to make path to there right answer short. Some constraints (for example, disequality~\cite{WillThesis}) are universally applicable, but it's OK to invent new special constraints for specific tasks.

The \miniKanren{} family of languages includes different implementations, both statically and dynamically typed. There are some peculiarities in statically typed implementation \OCanren{}~\cite{ocanren} relatively to ``official'' implementation~\cite{fasterMK}. For example, the following result of the query is decent in languages like
\Scheme{}, but in statically typed \OCaml{} the result looks weird.

\begin{minipage}{7cm}
\begin{lstlisting}{language=scheme}
; Scheme
> (run 1 (q)  (=/= q #t) (=/= q #f) )
((_.0 (=/= ((_.0 #f)) ((_.0 #t)))))
\end{lstlisting}
\end{minipage}

\begin{minipage}{9.5cm}
\begin{lstlisting}{language=ocaml}
(* OCaml *)
> run ... (funq->(q =/= !!true) &&& (q =/= !!false))
q=_.0 [=/= false; =/= true];
\end{lstlisting}
\end{minipage}

\noindent Indeed, in \Scheme{} there are infinitely many possible values for variables that are neither \lstinline[]=#t=, nor
\lstinline[]=#f=. But in \OCanren{} the compiler prohibits unification of variables with non-unifiable types, so variable \verb=q= could be bound in a substitution only to \lstinline=true=, \lstinline=false= or another fresh variable; and expected result of the query is empty stream.

The example above could be repaired by introduction of finite domain constraints~\cite{cKanren}, but in case of proper algebraic data types they doesn't help. Let's imagine that we want to express that \emph{a variable holds a list, but couldn't begin from a constructor \lstinline=Cons=}. The na\"{i}ve attempt in \OCanren{}, \lstinline|fresh (h tl) (q=/= cons h tl) | doesn't give us what we desire. It states that \empty{there are some} \lstinline|h| and \lstinline|tl| such that \lstinline|q| is not equal, but we expected that fact for any possible \lstinline|h| and \lstinline|tl|. This form of universal quantification is currently not expressible in \OCanren{}.

Another example of algebraic data types are Peano numbers. Unification allows us to express that a peano number \lstinline|q| is greater or equal a constant: \lstinline|q === S(S(S _.10))|$\!\!$.
But disequality constraints are not powerful enough to express that a number is \emph{less} than a constant.

In this paper we introduce \emph{wildcard logic variables
} (denoted as \lstinline|__|) which are able to solve problem like above.
The disequality \lstinline|q =/= S(S(S __))| states that two values are not equal no matter what we would substitute instead of \lstinline|__|, which will effectively filter out \lstinline|S(S(S Z))|, \lstinline|S(S(S(S Z)))|, i.e. all numbers greater or equal three. This \emph{single} disequality is an only constraint that is required  to describe  finitely ``a peano number is less then constant N''.
In default implementations of \miniKanren{} we could write a disjunction of three cases but this will hurt performance of the search.



% !TEX TS-program = pdflatex
% !TeX spellcheck = en_US
% !TEX root = main.tex

\section{Informal Description}
\label{sec:theory}

In this section we describe the essence of wildcard variables and how they interact with other features of \miniKanren{}.

\subsection{Wildcards and Disequality Constraints}
Checking and storing disequality constraints may be performed in clever manner~\cite{fasterMK}.
%But here we will use a na\"{i}ve implementation for simplicity.
Disequality constraints are stored as a conjunction of disjunctions of pairs (CNF): a fresh variable and a term (or an another variable).
While the search  progresses, new bindings are propagated to inequalities which allows simplification.
For example, if one of the bindings is impossible to be unified, then it is being removed from the disjunction clause.
If whole disjunction  becomes non-unifiable, then it could be removed from CNF.
If a disjunction clause becomes unifiable in any substitution, then it becomes violated and whole CNF too.
%this part of constraints couldn't provide an inequality and should be removed from a disjunction clause. If disjunction clause becomes empty, then constraints are violated and we should prune this branch of search tree.

In our implementation wildcard variables may occur in program many times. But internally it is a single logic variable (predefined ``constant'' is some sense), that it treated in a special way. On creation of a disequality constraint we perform unification which gives an updated substitution and a list of recently introduced bindings. Our implementation of wildcard unification doesn't add anything to substitution, but adds new bindings as usual. Our disequality constraints implementation evaluates these bindings and stores in every conjunction clause not only pairs, but also a set of variables that should not be wildcards.

Let's discuss details of checking disequality constraints in presence of wildcards using examples. Below we will use the mantra \emph{``It should be a way to make two values not equal, in spite of we could substitute anything instead of wildcard.''} to make decisions about simplification of constraints. We start from simple cases, and leave the complicated cases (a disequality between fresh variable and wildcard) to the end.

Consider the disequality \lstinline|(1 _.10)=/= (__ 2)| where either first components of pairs should be not equal, or second ones. With first components we are going to consider worst case scenario and substitute a number 1 instead of wildcard. The constraint above simplifies to a shorter disequality between  \lstinline|_.10| and \lstinline|2|. All disequalities between ground values and wildcard variable are simplified immediately.

On disequality of two wildcard variables we again consider worst case scenario: we substitute, for example, \lstinline|42| instead of both and get a violation of disequality constraints.

Disequality between wildcard and complex value, for example  \lstinline|cons 1 _.10=/= __|, could be also simplified. We consider a worst case scenario and use \lstinline|cons __ __| instead of wildcard. (But this simplification requires deep understanding of variable's domain, and currently not implemented.)

Disequalities between fresh variables and complex values with wildcards inside are left as they are. Later, we could get more information  about fresh variable and simplify the constraint. For example,
 \lstinline|(cons __ __ =/= _.10) & (cons _.11 _.12 === _.10)| simplifies to ``either\lstinline| _.11|  or \lstinline|_.12| is not wildcard''.

The case above leads us to the most complicated case: a disequality between fresh variable and wildcard. If we consider fresh variables as existential ones, the decision may look trivial: if variable \lstinline|q| should not be equal wildcard, we will easily violate this constraint by substituting \lstinline|q| instead of wildcard. But actually this substitution could be not possible, for example, if variable \lstinline|q| has an empty domain. Moreover, we should use hypothesis that any fresh variables have non empty domain, to allow attaching a domain information to variable \lstinline|q| later in the search. Without this hypothesis our \miniKanren{} implementation would be less declarative.



\subsection{Unification}
The primary usage of wildcard variables is in the context of disequality constraints.
Despite there is only a single wildcard variable in runtime, we  need a special combinator (similar to \lstinline|call_fresh|) that creates wildcard variable.
This is specific to \OCanren{} -- typed embedding of \miniKanren{} to \OCaml{} -- where we can unify only logic values of the same type. That's why we need many instances of differently typed wildcard variables.

The proposed wildcard variables are not designed to be used in unification, but we decided to allow wildcard syntax \lstinline|__| in unification anyway.
This ``wildcards'' have a different semantics from proposed in this paper, they looks resembles more traditional wildcards from pattern matching in \OCaml{}: they are just a convenient syntax to avoid manual creation of fresh variables. In the example below we demonstrate two implementations of a relation that extract tail of the list: with wildcard syntax and without them. All occurrences of \lstinline|__| are translated to calls of primitive combinators by a macro expansion.

\begin{minipage}{\linewidth}
\begin{lstlisting}{language=ocaml}
let tlo xs tl =
  (xs === List.conso __ tl)

let tlo xs tl =
  fresh (h) (xs === List.conso h tl)
\end{lstlisting}
\end{minipage}

\noindent The macro expansion is responsible for insertion of wildcard variables into right places, but end user could create meaningless \OCanren{} expression bypassing macro expansion. Right now we don't defend against that. But it should be doable if we add for every logic variable a phantom type variable\footnote{\href{https://kcsrk.info/ocaml/types/2016/06/30/behavioural-types/}{Behavioural types} by KC Sivaramakrishnan (accessed: August 29, 2022)} which says whether logic variable can be used in unification, disequality or both.

% !TeX spellcheck = en_US
% !TEX root = main.tex

\section{Matching}
\label{sec:matching}

%
% !TeX spellcheck = en_US
% !TEX root = main.tex

\section{``Technical meat''}
\label{sec:tech}

% !TeX spellcheck = en_US
% !TEX root = main.tex

\section{Related works}
\label{sec:related}

% !TeX spellcheck = en_US
% !TEX root = main.tex

\section{Conclusion and future work}

%We presented an approach for pattern matching implementation synthesis using relational programming. Currently, it demonstrates a good performance only
%on a very small problems. The performance can be improved by searching for new ways to prune the search space and by speeding up the implementation of
%relations and structural constraints. Also it could be interesting to integrate structural constraints more closely into \textsc{OCanren}'s core.
%Discovering an optimal order of samples and reducing the complete set of samples is another direction for research.
%
%The language of intermediate representation can be altered, too. It is interesting to add to an intermediate language so-called \emph{exit nodes}
%described in~\cite{maranget2001}. The straightforward implementation of them might require nominal unification, but we are not aware of any
%\textsc{miniKanren} implementation in which both disequality constraints and nominal unification~\cite{alphaKanren} coexist nicely.
%
%At the moment we support only simple pattern matching without any extensions. It looks technically easy to extend our approach with
%non-linear and disjunctive patterns. It will, however, increase the search space and might require more optimizations.
%
%
%


\section{Appendix}
%%% Autogenerated 06 Jul, 2020 17:40:19

\begin{lstlisting}
(* A|B|C *)
match ... with
| A -> 1 
| B -> 1 
| C -> 0 
\end{lstlisting}

\begin{lstlisting}
(* BIG 2 clauses  *)
match ... with
| triple (_, _, cons (Ldi (_), _)) -> 1 
| triple (_, _, cons (Push, _)) -> 2 
\end{lstlisting}

\begin{lstlisting}
(* BIG : 2.5 clauses  *)
match ... with
| triple (_, _, cons (Ldi (_), _)) -> 1 
| triple (_, _, cons (Push, _)) -> 2 
| triple (Int (_), _, cons (IOp (_), _)) -> 3 
\end{lstlisting}

\begin{lstlisting}
(* bool *)
match ... with
| true -> 1 
| false -> 0 
\end{lstlisting}

\begin{lstlisting}
(* bool*bool *)
match ... with
| pair (true, _) -> 1 
| pair (_, true) -> 1 
| pair (false, false) -> 0 
\end{lstlisting}

\begin{lstlisting}
(* bool*bool*bool (Maranget;page1) *)
match ... with
| triple (_, false, true) -> 1 
| triple (false, true, _) -> 2 
| triple (_, _, false) -> 3 
| triple (_, _, true) -> 4 
\end{lstlisting}

\begin{lstlisting}
(* simple nats (a la Maranget2008) *)
match ... with
| pair (succ (_), succ (_)) -> 30 
| pair (zero, _) -> 10 
| pair (_, zero) -> 10 
\end{lstlisting}

\begin{lstlisting}
(* two-nil lists (with cons; simplified RHS) *)
match ... with
| pair (nil, _) -> 10 
| pair (_, nil) -> 10 
| pair (nil2, _) -> 10 
| pair (_, nil2) -> 10 
| pair (cons (_, _), cons (_, _)) -> 60 
\end{lstlisting}



\begin{comment}
%% Acknowledgments
\begin{acks}                            %% acks environment is optional
                                        %% contents suppressed with 'anonymous'
  %% Commands \grantsponsor{<sponsorID>}{<name>}{<url>} and
  %% \grantnum[<url>]{<sponsorID>}{<number>} should be used to
  %% acknowledge financial support and will be used by metadata
  %% extraction tools.
  This material is based upon work supported by the
  \grantsponsor{GS100000001}{Russian Foundation for Basic Research}{https://www.rfbr.ru/rffi/eng} under Grant
  No.~\grantnum{GS100000001}{18-01-00380} and by the grant from JetBrains Research.
  %Any opinions, findings, and
  %conclusions or recommendations expressed in this material are those
  %of the author and do not necessarily reflect the views of the
  %National Science Foundation.
\end{acks}
\end{comment}

\bibliography{references}

\end{document}
