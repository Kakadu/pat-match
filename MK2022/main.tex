% !TEX TS-program = pdflatex
% !TeX spellcheck = en_US
% !TEX root = main.tex
\documentclass
  [screen
  , sigplan
  %, acmlarge
  %, draft
  , review
  ]{acmart}

\usepackage[
    type={CC},           % your choice
    modifier={by-sa},    % your choice
    version={4.0},       % your choice
]{doclicense}            % your choice, see \doclicenseThis below


\AtBeginDocument{%
  \providecommand\BibTeX{{%
    Bib\TeX}}}

%% Rights management information.  This information is sent to you
%% when you complete the rights form.  These commands have SAMPLE
%% values in them; it is your responsibility as an author to replace
%% the commands and values with those provided to you when you
%% complete the rights form.
\setcopyright{acmcopyright}
\copyrightyear{2022}
\acmYear{2022}
\acmDOI{XXXXXXX.XXXXXXX}

%% These commands are for a PROCEEDINGS abstract or paper.
\acmConference[miniKanren 2022]{The Fourth miniKanren and Relational Programming Workshop}{September 15, 2022}{Ljubljana, Slovenia}
\acmPrice{15.00}
\acmISBN{978-1-4503-XXXX-X/18/06}
\acmMonth{9}
\acmArticle{1} % your number in the order of presentations (between 1 and 11)

\usepackage{lineno}
%\linenumbers
% !TeX spellcheck = en_US
% !TEX root = main.tex

\usepackage{alltt}
\usepackage{latexsym}
\usepackage{array}
\usepackage{comment}
%\usepackage{makeidx}
%\usepackage{indentfirst}
%\usepackage{verbatim}
%\usepackage{color}
%\usepackage{url}
%\usepackage{xspace}
%\usepackage{hyperref}
%\usepackage{stmaryrd}
\usepackage{amsmath, amsthm}
%\usepackage{graphicx}
%\usepackage{euscript}
\usepackage{mathtools}
%\usepackage{mathrsfs}
%\usepackage{multirow,bigdelim}
%\usepackage{subcaption}
%\usepackage{placeins}
\usepackage{csvsimple}
\usepackage{array}

%% Some recommended packages.
\usepackage{booktabs}   %% For formal tables:
                       %% http://ctan.org/pkg/booktabs
\usepackage{subcaption} %% For complex figures with subfigures/subcaptions
                       %% http://ctan.org/pkg/subcaption
\usepackage{multirow}

\usepackage{placeins}

\usepackage{listings}
\lstdefinelanguage{ocanren}{
keywords={run, conde, fresh, let, in, match, with, when, class, type,
object, method, of, rec, repeat, until, while, not, do, done, as, val, inherit,
new, module, sig, deriving, datatype, struct, if, then, else, open, private, virtual, include, success, failure,
true, false},
sensitive=true,
commentstyle=\small\itshape\ttfamily,
keywordstyle=\ttfamily\textbf,
identifierstyle=\ttfamily,
basewidth={0.5em,0.5em},
columns=fixed,
mathescape=true,
fontadjust=true,
literate={fun}{{$\lambda$}}1 {->}{{$\to$}}3 {===}{{$\equiv$}}1 {=/=}{{$\not\equiv$}}1 {|>}{{$\triangleright$}}3 {\\/}{{$\vee$}}2 {/\\}{{$\wedge$}}2 {^}{{$\uparrow$}}1,
morecomment=[s]{(*}{*)}
}

\lstset{
%mathescape=true,
%basicstyle=\small,
%identifierstyle=\ttfamily,
%keywordstyle=\bfseries,
%commentstyle=\scriptsize\rmfamily,
%basewidth={0.5em,0.5em},
%fontadjust=true,
language=ocanren
}

\newcommand{\lstquot}[1]{``\lstinline{#1}''}
\newcommand{\sembr}[1]{\llbracket{#1}\rrbracket}
\newcommand\false{$f\!alse$}
\newcommand\myif{i\!f}


\def\transarrow{\xrightarrow}
\newcommand{\setarrow}[1]{\def\transarrow{#1}}

\def\padding{\phantom{X}}
\newcommand{\setpadding}[1]{\def\padding{#1}}

\def\subarrow{}
\newcommand{\setsubarrow}[1]{\def\subarrow{#1}}

\newcommand{\OCaml}{\textsc{OCaml}}
\newcommand{\OCanren}{\textsc{OCanren}}
\newcommand{\miniKanren}{\textsc{miniKanren}}
\newcommand{\Scheme}{\textsc{Scheme}}

\newcommand{\trule}[2]{\dfrac{#1}{#2}}
\newcommand{\crule}[3]{\dfrac{#1}{#2},\;{#3}}
\newcommand{\withenv}[2]{{#1}\vdash{#2}}
\newcommand{\trans}[3]{{#1}\transarrow{\padding{\textstyle #2}\padding}\subarrow{#3}}
\newcommand{\ctrans}[4]{{#1}\transarrow{\padding#2\padding}\subarrow{#3},\;{#4}}
\newcommand{\llang}[1]{\mbox{\lstinline[mathescape]|#1|}}
\newcommand{\pair}[2]{\inbr{{#1}\mid{#2}}}
\newcommand{\inbr}[1]{\left<{#1}\right>}
\newcommand{\highlight}[1]{\color{red}{#1}}
%\newcommand{\ruleno}[1]{\eqno[\scriptsize\textsc{#1}]}
\newcommand{\ruleno}[1]{\mbox{[\textsc{#1}]}}
\newcommand{\rulename}[1]{\textsc{#1}}
\newcommand{\inmath}[1]{\mbox{$#1$}}
\newcommand{\lfp}[1]{fix_{#1}}
\newcommand{\gfp}[1]{Fix_{#1}}
\newcommand{\vsep}{\vspace{-2mm}}
\newcommand{\supp}[1]{\scriptsize{#1}}
\renewcommand{\sembr}[1]{\llbracket{#1}\rrbracket}
\newcommand{\cd}[1]{\texttt{#1}}
\newcommand{\free}[1]{\boxed{#1}}
\newcommand{\binds}{\;\mapsto\;}
\newcommand{\dbi}[1]{\mbox{\bf{#1}}}
\newcommand{\sv}[1]{\mbox{\textbf{#1}}}
\newcommand{\bnd}[2]{{#1}\mkern-9mu\binds\mkern-9mu{#2}}
\newcommand{\meta}[1]{{\mathcal{#1}}}
\newcommand{\dom}[1]{\mathtt{dom}\;{#1}}
%\newcommand{\primi}[2]{\mathbf{#1}\;{#2}}
\renewcommand{\dom}[1]{\mathcal{D}om\,({#1})}
\newcommand{\ran}[1]{\mathcal{VR}an\,({#1})}
\newcommand{\fv}[1]{\mathcal{FV}\,({#1})}
\newcommand{\tr}[1]{\mathcal{T}r_{#1}}
\newcommand{\diseq}{\not\equiv}
\newcommand{\reprfunset}{\mathcal{R}}
\newcommand{\reprfun}{\mathfrak{f}}
\newcommand{\cstore}{\Omega}
\newcommand{\cstoreinit}{\cstore_\epsilon^{init}}
\newcommand{\csadd}[3]{add(#1, #2 \diseq #3)}  %{#1 + [#2 \diseq #3]}
\newcommand{\csupdate}[2]{update(#1, #2)}  %{#1 \cdot #2}
\newcommand{\primi}[1]{\mathbf{#1}}
\newcommand{\sem}[1]{\llbracket #1 \rrbracket}
\newcommand{\ir}{\ensuremath{\mathcal{S}}}
\usepackage{tikz}
\newcommand*\circled[1]{\tikz[baseline=(char.base)]{
   \node[shape=circle,draw,inner sep=1pt] (char) {#1};}}

\let\emptyset\varnothing
\let\eps\varepsilon

\newtheorem{Observation}{Observation}

\newcommand{\todo}[1]{\textcolor{blue}{#1}}
%\geometry{
%     top=18pt, bottom=14pt, inner=21pt, outer=21pt,
%     paperwidth=5.5in, paperheight=8.5in,
%     }

\settopmatter{printacmref=false,printfolios=false}
\fancyfoot{}

\makeatletter
\def\@formatdoi#1{}
\def\@permissionCodeOne{miniKanren.org/workshop}
\def\@copyrightpermission{\doclicenseThis} % your choice of text
\def\@copyrightowner{Copyright held by the author(s).} % your choice
\makeatother

\copyrightyear{2022}
\setcopyright{rightsretained}


\acmConference[miniKanren 2022]{The Fourth miniKanren and Relational Programming Workshop}{September 15 2022}{Ljubljana, Slovenia}

\acmMonth{8}
\acmArticle{8} % your number in the order of presentations (between 1 and 11)



%% Bibliography style
\bibliographystyle{ACM-Reference-Format}
%% Citation style
%% Note: author/year citations are required for papers published as an
%% issue of PACMPL.
\citestyle{acmauthoryear}   %% For author/year citations


%%%%%%%%%%%%%%%%%%%%%%%%%%%%%%%%%%%%%%%%%%%%%%%%%%%%%%%%%%%%%%%%%%%%%%
%% Note: Authors migrating a paper from PACMPL format to traditional
%% SIGPLAN proceedings format must update the '\documentclass' and
%% topmatter commands above; see 'acmart-sigplanproc-template.tex'.
%%%%%%%%%%%%%%%%%%%%%%%%%%%%%%%%%%%%%%%%%%%%%%%%%%%%%%%%%%%%%%%%%%%%%%

\sloppy


\begin{document}

\title{Wildcard Logic Variables\\
  \large How to say in \miniKanren{} that natural number is less than 5? }


%\titlenote{This work was partially supported by the grant 18-01-00380 from The Russian Foundation for Basic Research} %% \titlenote is optional;

%\author{John Doe}
%\affiliation{
%	\institution{Ice Crown University}
%	\country{Northrend}
%}
%\email{lichkind@wow.com}


%\author{Dmitry Kosarev}
%\affiliation{
%  \institution{Saint Petersburg State University}
%  \country{Russia}
%}
%\affiliation{
%  \institution{JetBrains Research}
%  \country{Russia}
%}
%\email{Dmitrii.Kosarev@pm.me}

%\author{Dmitry Boulytchev}
%\affiliation{
%	\institution{Saint Petersburg State University}
%	\country{Russia}
%}
%
%\affiliation{
%	\institution{JetBrains Research}
%	\country{Russia}
%}
%\email{dboulytchev@math.spbu.ru}

%% Abstract
%% Note: \begin{abstract}...\end{abstract} environment must come
%% before \maketitle command
\begin{abstract}
We propose a new kind of logic variables -- wildcard variables -- as a limited form of universal quantification.
Combined with disequality constraints they extend the expressive power of \OCanren{} -- typed dialect of \miniKanren{}, and enrich subset of \OCaml{} programs that could be automatically converted to relational ones. We also report our progress on applying this idea to a task of synthesizing pattern matching compilation scheme.
\end{abstract}


%% 2012 ACM Computing Classification System (CSS) concepts
%% Generate at 'http://dl.acm.org/ccs/ccs.cfm'.
\begin{CCSXML}
<ccs2012>
<concept>
<concept_id>10011007.10011006.10011008.10011009.10011015</concept_id>
<concept_desc>Software and its engineering~Constraint and logic languages</concept_desc>
<concept_significance>500</concept_significance>
</concept>
<concept>
<concept_id>10011007.10011006.10011041.10011047</concept_id>
<concept_desc>Software and its engineering~Source code generation</concept_desc>
<concept_significance>500</concept_significance>
</concept>
</ccs2012>
\end{CCSXML}

\ccsdesc[500]{Software and its engineering~Constraint and logic languages}
\ccsdesc[500]{Software and its engineering~Source code generation}
%% End of generated code


%% Keywords
%% comma separated list
\keywords{relational programming, relational interpreters, pattern matching, constraint programming}  
%% \keywords are mandatory in final camera-ready submission


%% \maketitle
%% Note: \maketitle command must come after title commands, author
%% commands, abstract environment, Computing Classification System
%% environment and commands, and keywords command.

\maketitle

\thispagestyle{empty}

% !TEX TS-program = pdflatex
% !TeX spellcheck = en_US
% !TEX root = main.tex
\section{Introduction}
\label{sec:intro}

Relational and logic programming are powerful techniques for enumerating the space of possible answers for a query. Constraints allow us to prune search space and to make path to there right answer short. Some constraints (for example, disequality~\cite{WillThesis}) are universally applicable, but it's OK to invent new special constraints for specific tasks.

The \miniKanren{} family of languages includes different implementations, both statically and dynamically typed. There are some peculiarities in statically typed implementation \OCanren{}~\cite{ocanren} relatively to ``official'' implementation~\cite{fasterMK}. For example, the following result of the query is decent in languages like
\Scheme{}, but in statically typed \OCaml{} the result looks weird.

\begin{minipage}{7cm}
\begin{lstlisting}{language=scheme}
; Scheme
> (run 1 (q)  (=/= q #t) (=/= q #f) )
((_.0 (=/= ((_.0 #f)) ((_.0 #t)))))
\end{lstlisting}
\end{minipage}

\begin{minipage}{9.5cm}
\begin{lstlisting}{language=ocaml}
(* OCaml *)
> run ... (funq->(q =/= !!true) &&& (q =/= !!false))
q=_.0 [=/= false; =/= true];
\end{lstlisting}
\end{minipage}

\noindent Indeed, in \Scheme{} there are infinitely many possible values for variables that are neither \lstinline[]=#t=, nor
\lstinline[]=#f=. But in \OCanren{} the compiler prohibits unification of variables with non-unifiable types, so variable \verb=q= could be bound in a substitution only to \lstinline=true=, \lstinline=false= or another fresh variable; and expected result of the query is empty stream.

The example above could be repaired by introduction of finite domain constraints~\cite{cKanren}, but in case of proper algebraic data types they doesn't help. Let's imagine that we want to express that \emph{a variable holds a list, but couldn't begin from a constructor \lstinline=Cons=}. The na\"{i}ve attempt in \OCanren{}, \lstinline|fresh (h tl) (q=/= cons h tl) | doesn't give us what we desire. It states that \empty{there are some} \lstinline|h| and \lstinline|tl| such that \lstinline|q| is not equal, but we expected that fact for any possible \lstinline|h| and \lstinline|tl|. This form of universal quantification is currently not expressible in \OCanren{}.

Another example of algebraic data types are Peano numbers. Unification allows us to express that a peano number \lstinline|q| is greater or equal a constant: \lstinline|q === S(S(S _.10))|$\!\!$.
But disequality constraints are not powerful enough to express that a number is \emph{less} than a constant.

In this paper we introduce \emph{wildcard logic variables
} (denoted as \lstinline|__|) which are able to solve problem like above.
The disequality \lstinline|q =/= S(S(S __))| states that two values are not equal no matter what we would substitute instead of \lstinline|__|, which will effectively filter out \lstinline|S(S(S Z))|, \lstinline|S(S(S(S Z)))|, i.e. all numbers greater or equal three. This \emph{single} disequality is an only constraint that is required  to describe  finitely ``a peano number is less then constant N''.
In default implementations of \miniKanren{} we could write a disjunction of three cases but this will hurt performance of the search.




% \section{TODO}
%   \begin{todolist}
%     \item[\done] Intro
%     \item[\wip] Theory
%         \begin{todolist}
%           \item Add many examples to appendix 
%         \end{todolist}
%     \item[\wip] Related 
%     \begin{todolist}
%       \item[\wip] Eigen 
%         \begin{itemize}
%           \item Add  examples for comparison
%         \end{itemize}
%       \item[\wip] Universal quantification and implication (???)
%       \item[\wip] noCanren 
%       \item[\wip] Matching 
%     \end{todolist}
%     \item[\done] Application to noCanren 
%     \item Application to pattern matching 
%         \begin{todolist}
%           \item More formal description
%           \item Proposed encoding of pattern 
%           \item Why domains are important?
%           \item WTF it doesn't work
%         \end{todolist}
%      \item ``Technical meat''   
%      \item Conclusion
%   \end{todolist}
  
% !TEX TS-program = pdflatex
% !TeX spellcheck = en_US
% !TEX root = main.tex

\section{Informal Description}
\label{sec:theory}

In this section we describe the essence of wildcard variables and how they interact with other features of \miniKanren{}.

\subsection{Wildcards and Disequality Constraints}
Checking and storing disequality constraints may be performed in clever manner~\cite{fasterMK}.
%But here we will use a na\"{i}ve implementation for simplicity.
Disequality constraints are stored as a conjunction of disjunctions of pairs (CNF): a fresh variable and a term (or an another variable).
While the search  progresses, new bindings are propagated to inequalities which allows simplification.
For example, if one of the bindings is impossible to be unified, then it is being removed from the disjunction clause.
If whole disjunction  becomes non-unifiable, then it could be removed from CNF.
If a disjunction clause becomes unifiable in any substitution, then it becomes violated and whole CNF too.
%this part of constraints couldn't provide an inequality and should be removed from a disjunction clause. If disjunction clause becomes empty, then constraints are violated and we should prune this branch of search tree.

In our implementation wildcard variables may occur in program many times. But internally it is a single logic variable (predefined ``constant'' is some sense), that it treated in a special way. On creation of a disequality constraint we perform unification which gives an updated substitution and a list of recently introduced bindings. Our implementation of wildcard unification doesn't add anything to substitution, but adds new bindings as usual. Our disequality constraints implementation evaluates these bindings and stores in every conjunction clause not only pairs, but also a set of variables that should not be wildcards.

Let's discuss details of checking disequality constraints in presence of wildcards using examples. Below we will use the mantra \emph{``It should be a way to make two values not equal, in spite of we could substitute anything instead of wildcard.''} to make decisions about simplification of constraints. We start from simple cases, and leave the complicated cases (a disequality between fresh variable and wildcard) to the end.

Consider the disequality \lstinline|(1 _.10)=/= (__ 2)| where either first components of pairs should be not equal, or second ones. With first components we are going to consider worst case scenario and substitute a number 1 instead of wildcard. The constraint above simplifies to a shorter disequality between  \lstinline|_.10| and \lstinline|2|. All disequalities between ground values and wildcard variable are simplified immediately.

On disequality of two wildcard variables we again consider worst case scenario: we substitute, for example, \lstinline|42| instead of both and get a violation of disequality constraints.

Disequality between wildcard and complex value, for example  \lstinline|cons 1 _.10=/= __|, could be also simplified. We consider a worst case scenario and use \lstinline|cons __ __| instead of wildcard. (But this simplification requires deep understanding of variable's domain, and currently not implemented.)

Disequalities between fresh variables and complex values with wildcards inside are left as they are. Later, we could get more information  about fresh variable and simplify the constraint. For example,
 \lstinline|(cons __ __ =/= _.10) & (cons _.11 _.12 === _.10)| simplifies to ``either\lstinline| _.11|  or \lstinline|_.12| is not wildcard''.

The case above leads us to the most complicated case: a disequality between fresh variable and wildcard. If we consider fresh variables as existential ones, the decision may look trivial: if variable \lstinline|q| should not be equal wildcard, we will easily violate this constraint by substituting \lstinline|q| instead of wildcard. But actually this substitution could be not possible, for example, if variable \lstinline|q| has an empty domain. Moreover, we should use hypothesis that any fresh variables have non empty domain, to allow attaching a domain information to variable \lstinline|q| later in the search. Without this hypothesis our \miniKanren{} implementation would be less declarative.



\subsection{Unification}
The primary usage of wildcard variables is in the context of disequality constraints.
Despite there is only a single wildcard variable in runtime, we  need a special combinator (similar to \lstinline|call_fresh|) that creates wildcard variable.
This is specific to \OCanren{} -- typed embedding of \miniKanren{} to \OCaml{} -- where we can unify only logic values of the same type. That's why we need many instances of differently typed wildcard variables.

The proposed wildcard variables are not designed to be used in unification, but we decided to allow wildcard syntax \lstinline|__| in unification anyway.
This ``wildcards'' have a different semantics from proposed in this paper, they looks resembles more traditional wildcards from pattern matching in \OCaml{}: they are just a convenient syntax to avoid manual creation of fresh variables. In the example below we demonstrate two implementations of a relation that extract tail of the list: with wildcard syntax and without them. All occurrences of \lstinline|__| are translated to calls of primitive combinators by a macro expansion.

\begin{minipage}{\linewidth}
\begin{lstlisting}{language=ocaml}
let tlo xs tl =
  (xs === List.conso __ tl)

let tlo xs tl =
  fresh (h) (xs === List.conso h tl)
\end{lstlisting}
\end{minipage}

\noindent The macro expansion is responsible for insertion of wildcard variables into right places, but end user could create meaningless \OCanren{} expression bypassing macro expansion. Right now we don't defend against that. But it should be doable if we add for every logic variable a phantom type variable\footnote{\href{https://kcsrk.info/ocaml/types/2016/06/30/behavioural-types/}{Behavioural types} by KC Sivaramakrishnan (accessed: August 29, 2022)} which says whether logic variable can be used in unification, disequality or both.

% !TEX TS-program = pdflatex
% !TeX spellcheck = en_US
% !TEX root = main.tex


\section{Related works}
\label{sec:related}

The described approach with wildcard logic variable
strongly reminds a form of universal quantification. 
Nowadays, the ``official'' implementation~\cite{fasterMK} of \miniKanren{} in \Scheme{} don't yet support any form of universal quantification. In this section we will observe two approaches two universal quantification, and two areas where our wildcard logic variables could be helpful.

\subsection{Eigen}

One of the form of universal quantification are eigen variables~\cite{eigen}. This approach allows to introduce new fresh named existential variables as usual, and new eigen named variables, which are unifiable with themselves and with fresh variables introduced in their scope. The primary purpose of eigen variables is synthesis a fixpoint combinator in combinatory logic, and this task is being solved without any usage of disequality or other constraints. The interaction between eigen variables and constraints is not handled by the implementation of eigen dialect of \miniKanren{}. During personal interactions over email we are told that the performance of current implementation of eigen and disequality constraints is troublesome, because it seems to be required to recheck possible equalities between all introduced logic variables and eigen variables. 

The one could consider our approach a simplified form of eigen variables. 
We don't attach any names to our wildcard variables, so rechecking all possible pairs of inequalities doesn't make much sense.
This makes out approach less expressive than eigen but definitely improves search termination.  
Say, from one point of view the following goal can't be expressed via wildcard variables.
\begin{lstlisting}[language=ocanren]
(run 1 (q) (fresh (a b) (eigen (x) 
    (=/= `(,x ,x) `(,a ,b))) (== a 7) (== b 7)))
\end{lstlisting}
From another point of view, this goal diverges since it calculation collects all disequality constraints and then simplifies process them once.

Next, eigen variable may occur before fresh variables while it is not the case with wildcards.
Consider the following example with eigen variable: 
\begin{lstlisting}[xleftmargin=0.5cm]
(run 1 (q) (eigen (x) (fresh (y) (=/= x y))))
\end{lstlisting}
It succeeds since whatever for any given \lstinline{x}, one can choose a \lstinline{y} distinct from \lstinline{x}.
%that's different than it.
While wildcarded expression 
\begin{lstlisting}[xleftmargin=1cm]
(run 1 (q) (fresh (y) (__ =/= y)))
\end{lstlisting}
fails as well as its eigen equivalent 
\begin{lstlisting}[xleftmargin=0.5cm]
(run 1 (q) (fresh (y) (eigen (x) (=/= x y))).
\end{lstlisting}

Also, our wildcard variables should be used in disequality constraints.
Antagonistically, the implementation of eigen variables uses them in unifications. 
The detailed comparison of expressiveness of wildcards and eigen variables in disequality constraint requires further studying. 
  
\subsection{Universal Quantification and Implication}

An interesting idea from \cite{universal2021} it to mine examples from the domain of universally quantified variable one by one, cut these points from the domain and wait until it becomes empty. This approach is promising for finite domains, but for recursively described ones it could lead to divergence. We are looking forward for upgraded implementations of the approach, to check it out for tasks that are important for us. 

\subsection{\noCanren{}}

Writing relational programs well requires gaining some skill. The tempting idea is to generate relational programs from functional ones. 
A subset of OCaml could be automatically converted~\cite{RelConversion} to \OCanren{}, but current restrictions on a language make the usage of it inconvenient. For example, to have a decent semantics of a relation programs, it's required for every pattern matching in functional program that all it's branches are non-overlapping. This shortcoming exists because of normal disequality constraints are not powerful enough to express desired result, and the manual process of making pattern non-overlapping leads to exponential increasing of a number of patterns. We believe that adding wildcard variables will solve this embarrassment (section~\ref{sec:noCanren}).

\subsection{Relational Synthesis of Pattern Matching}

The state of art approach to compile pattern matching in \OCaml{} to intermediate representation is translation to backtracking automaton~\cite{maranget2001}. In theory we could implement a relational interpreter of intermediate representation, and synthesize a compilation scheme that behave on a pack of examples as we desire~\cite{Kosarev2020}. In practice, a number of examples is finite but large. It depends more on a number of inhabitants of scrutinee's type until certain depth, than on a number of patterns in pattern matching. Ideally, we want to have for exhaustive pattern matching as many examples as we have branches. The wildcard variables allow us to make a step in that direction. But they we are currently far away from finishing that task, because of hidden complications of the task. We describe our progress in section~\ref{sec:matching}.



% !TEX TS-program = pdflatex
% !TeX spellcheck = en_US
% !TEX root = main.tex

\begin{figure*}[t!]
  \begin{subfigure}[t]{0.48\textwidth}
    \centering
    \begin{subfigure}{0.4\linewidth}
      \centering
      \renewcommand\thesubfigure{\alph{subfigure}1}
      \begin{lstlisting}[xleftmargin=.7cm,linewidth=3.5cm]
match x, y with
| T, _ -> 1
| _, _ -> 2
      \end{lstlisting}
      %\vskip18.5mm
      \caption{Pattern matching}
      \label{fig:match-example-small-functional}
    \end{subfigure}
    \hspace{.5cm}
    \begin{subfigure}{0.5\linewidth}
      \centering
      \addtocounter{subfigure}{-1}
      \renewcommand\thesubfigure{\alph{subfigure}2}
      \begin{lstlisting}[xleftmargin=1.7cm,linewidth=3.5cm]
if x 
then 1 
else 2
      \end{lstlisting}
      %\vskip8.5mm
      \caption{Optimal implementation}
      %\label{fig:matching-example3}
    \end{subfigure}
    \addtocounter{subfigure}{-1}
    \caption{Simple patterns matching example} 
    \label{fig:match-example-small}
  \end{subfigure}
  \begin{subfigure}[t]{0.48\textwidth}
    \begin{subfigure}{0.4\linewidth}
      \centering
      \renewcommand\thesubfigure{\alph{subfigure}1}
      \begin{lstlisting}
match x, y, z with
| _, F, T -> 1
| F, T, _ -> 2
| _, _, F -> 3
| _, _, T -> 4
      \end{lstlisting}
      \vskip4.5mm
      \caption{Pattern matching}
      \label{fig:matching-example1}
    \end{subfigure}
% \hspace{0.2cm}
    \begin{subfigure}{0.5\linewidth}
      \centering
      \addtocounter{subfigure}{-1}
      \renewcommand\thesubfigure{\alph{subfigure}2}
      \begin{lstlisting}
if y then
  if x then
    if z then 4 else 3
  else 2
else
  if z then 1 else 3
      \end{lstlisting}
      %\vskip13.5mm
      \caption{Optimal implementation}
      \label{fig:matching-example2}
    \end{subfigure}
    \addtocounter{subfigure}{-1}
    \caption{Pattern matching compilation example (from~\cite{maranget2001})} 
    \label{fig:match-example}
  \end{subfigure}
  \begin{subfigure}{0.49\textwidth}
    \centering
    \begin{lstlisting}
let naive_rel q rez =
  conde
    [ fresh (tmp) 
        (q === pair !!true tmp) 
        (rez === !!1)
    ; fresh (tmp l r) 
        (q =/= pair !!true tmp) 
        (q === pair l r)
        (rez === !!2) 
    ]
    \end{lstlisting}
    \caption{Na\"{i}ve compilation with disequality constraints}
    \label{fig:match-compilation-small-diseq}
  \end{subfigure}
  \begin{subfigure}{0.49\textwidth}
    \centering
    \begin{lstlisting}
let better_rel q rez =
  conde
    [ fresh () 
        (q === pair !!true __) 
        (rez === !!1)
    ; fresh () 
        (q =/= pair !!true __) 
        (rez === !!2) 
    ]
    \end{lstlisting}
    \caption{Better compilation with wildcards}
    \label{fig:match-compilation-small-wc}
  \end{subfigure}
  \caption{Pattern matching compilation example } 
  \label{fig:match-compilation-small}
\end{figure*}

\begin{figure*}[t]
%\centering
\begin{subfigure}{0.49\textwidth}
\begin{lstlisting}
let enc_with_diseq q rez =
  let _T = !!true in
  let _F = !!false in
  let w = Std.triple in 
  conde
    [ fresh (fresh1) (rez === !!1) 
        (q === w fresh1 _F _T)
    ; fresh (fresh1 x) (rez === !!2) 
        (q === w _F _T fresh1) 
        (q =/= w x _F _T)
    ; fresh (fresh1 fresh2 x y z) (rez === !!3) 
        (q === w fresh1 fresh2 _F) 
        (q =/= w x _F _T) 
        (q =/= w _F _T z)
    ; fresh (x y z fresh1 fresh2) (rez === !!4)
        (q =/= w x _F _T)
        (q =/= w _F _T z)
        (q =/= w x y _F)
        (q === w fresh1 fresh2 _T) ]
\end{lstlisting}
\caption{With disequality constraints}
\label{fig:matching-example3}
\end{subfigure}
%\vskip13.5mm
\begin{subfigure}{0.49\textwidth}
\begin{lstlisting}
let enc_with_wc q rez =
    let _T = !!true in
    let _F = !!false in
    let w = Std.triple in
    conde
      [ fresh () (rez === !!1) 
          (q === w __ _F _T)
      ; fresh () (rez === !!2) 
          (q === w _F _T __) 
          (q =/= w __ _F _T)
      ; fresh () (rez === !!3) 
          (q === w __ __ _F) 
          (q =/= w __ _F _T) 
          (q =/= w _F _T __)
      ; fresh () (rez === !!4)
          (q =/= w __ _F _T)
          (q =/= w _F _T __)
          (q =/= w __ __ _F)
          (q === w __ __ _T) ]
\end{lstlisting}
\caption{With wildcards}
%\label{fig:matching-example3}
\end{subfigure}
\caption{Two possible encodings of an example from Fig.~\ref{fig:match-example}}
\label{fig:maranget-example-compilation}
%\vskip13.5mm
\end{figure*}

\begin{figure*}[t!]
%\centering
\begin{subfigure}{0.49\textwidth}
\begin{lstlisting}[escapechar=|,numbers=left]
q=((_.13, false, true), 1);
q=((false, true, _.15), 2);
q=((_.13, _.14, false), 3);  |\label{line:three}|
q=((_.13 [=/= false; =/= _.22], _.14, true), 4);
q=((_.13 [=/= false], _.14, false), 3);
q=((_.13 [=/= _.22], _.14, true), 4);
q=((_.13, _.14 [=/= true], false), 3);
q=((_.13 [=/= _.22], _.14 [=/= true], true), 4);        |\label{line:eight}|
q=((_.13 [=/= false], _.14 [=/= false], true), 4);
q=((_.13, _.14 [=/= false], true), 4);                  |\label{line:ten}|
q=((_.13, _.14 [=/= false; =/= true], true), 4);
\end{lstlisting}
\caption{With disequality constraints}
%\label{fig:matching-example3}
\end{subfigure}
%\vskip13.5mm
\begin{subfigure}{0.49\textwidth}
\begin{lstlisting}
q=((_.13, false, true), 1);
q=((false, true, _.15), 2);


q=((_.13 [=/= false], _.14, false), 3);

q=((_.13, _.14 [=/= true], false), 3);
q=((_.13 [=/= false], _.14 [=/= false], true), 4);


q=((_.13, _.14 [=/= false; =/= true], true), 4);
\end{lstlisting}
\caption{With wildcards}
%\label{fig:matching-example3}
\end{subfigure}
\caption{The output of running two encodings from Figure~\ref{fig:maranget-example-compilation} where scrutinee is a triple of three fresh variables \lstinline|(_.13, _.14, _.15)|. One can observe that disequality constraints generate more (bogus) answers. Also, the last answer raises the idea that finite domain constraints may be useful for that example.}
%\vskip13.5mm
\label{fig:matching-maranget-compilation}
\end{figure*}

\section{\noCanren{}}
\label{sec:noCanren}

Translation of pattern matching from functional program to relational \lstinline|conde| clauses could be non-trivial if a few branches overlap. 
For a functional program from Figure~\ref{fig:match-example-small-functional} one could try a straightforward encoding without any constraints:

\begin{lstlisting}
let straightforward q rez =
  conde
    [ fresh () 
        (rez === !!1)
        (q === Std.pair !!true __) 
    ; fresh (l r)  
        (rez === !!2)
        (q === Std.pair l r)
    ]
\end{lstlisting}

\noindent Unfortunately, this encoding to relational program will have a different semantics from functional one. If scrutinee is a pair of \lstinline|true|'s, the stream of answers will have both \lstinline|(rez===1)| and \lstinline|(rez===2)| from the first and the second branches of \lstinline|conde|, respectively. The first answer is expected, but the second one contradicts behaviour of the functional program: the second branch should not be tested if previous one fits the scrutinee.
This issue appears only if branches overlap, and to get a proper semantics the approach of relational conversion~\cite{RelConversion} and associated tool\footnote{\url{https://github.com/Lozov-Petr/noCanren} (accessed: August 1st, 2022)} decided to forbid overlapping branches and not to use any constraints in generated code. 
The developer needs to manually rewrite branches of pattern matching, which is annoying and could lead to exponential increasing of the number of branches.

For an example shown in Figure~\ref{fig:match-example-small} we also provide encodings (Figure~\ref{fig:match-compilation-small}) with default disequality constraints and with wildcards.

An execution of this pattern matching on four possible ground scrutinees, encoding using wildcards provides four expected results: if first element of pair is \lstinline{true}, return 1; on \lstinline{false} -- 2.
Running na\"ive compilation scheme (Figure~\ref{fig:match-compilation-small-diseq}) discovers two more answers: pairs \lstinline|(true,false)| and \lstinline|(true,true)| may return 2.
Indeed, two branches of conde are not ordered, and running this scrutinees will not fail a disequality constraint: it will be simplified to \lstinline|=/= true| and \lstinline|=/= false| respectively.
From this example one could conclude that disequality constraints are not expressive enough to handle relational conversion of pattern matching.

One also could try to encode a more complicated example from~\cite{maranget2001} with four branches and a scrutinee being a triple of booleans (Figure~\ref{fig:match-example}).
In Figure~\ref{fig:maranget-example-compilation} one could observe relational encoding of this matching, 
and in Figure~\ref{fig:matching-maranget-compilation} the result of querying where a scrutinee is a triple of three fresh variables.
One could see that all answers returned by ``wildcardful'' relation are also returned by a relation with disequality constraints.
Extra answers on the left should be considered bogus.
For example, answer in line~\ref{line:three} about branch 3 is incorrect: the described scrutinee should be already matched by branch 2.
Also, answers in lines~\ref{line:eight} and~\ref{line:ten} should subsume each other.
Finally, the last answer demonstrates the need of finite domain constraints which allows considering disequality constraint containing ambivalent information such as \lstinline{[=/= false; =/= true]} violated.

The above demonstrates how \emph{wildcard} usage allows one to compile pattern matching in a very elegant way without the need of disjoint \lstinline{conde}-s.

% !TEX TS-program = pdflatex
% !TeX spellcheck = en_US
% !TEX root = main.tex

\begin{figure*}[t]
\begin{lstlisting}[numbers=left]
q=(if S[0] then 4 else (if S[2] then (if S[1] then 2 else 1) else 3));
q=(if S[0] then 4 else (if S[2] then (if S[1] then 2 else 1) else (if S[1] then 3 else _.1494)));
q=(if S[0] then 4 else (if S[2] then 1 else (if S[1] then 2 else 3)));
q=(if S[0] then 4 else (if S[2] then (if S[1] then _.25863 else 1) else (if S[1] then 2 else 3)));
q=(if S[0] then 4 else (if S[1] then (if S[2] then 2 else 3) else 1));
q=(if S[0] then 4 else (if S[2] then (if S[1] then 2 else 1) else (if S[1] then _.1493 else 3)));
q=(if S[0] then 4 else (if S[1] then (if S[2] then 2 else 3) else (if S[2] then 1 else _.35148)));
q=(if S[0] then 4 else (if S[1] then 2 else (if S[2] then 1 else 3)));
q=(if S[0] then (if S[1] then 4 else 3) else (if S[1] then 2 else 1));
q=(if S[0] then (if S[1] then 4 else 3) else (if S[1] then 2 else (if S[2] then 1 else _.99286)));
\end{lstlisting}
\caption{First ten unexpected results while compiling pattern matching from Figure~\ref{fig:matching-example1}}
\label{fig:matching-result-wierd}
\end{figure*}

\section{Synthesis of Pattern Matching}
\label{sec:matching}

In this section we briefly describe the task of synthesis of pattern matching compilation scheme~\cite{Kosarev2020} and how wildcard patterns may improve the situation. 

The pattern matching expression match a scrutinee $v$ from a set of values $\mathcal{V}$ of algebraic data types with a finite set of patterns $\mathcal{P}$, and produces an integer -- index of the pattern, such that it matches provided scrutinee and the ones before it doesn't. For simplicity we suppose that the set patterns is exhaustive, i.e. it is impossible to provide the scrutinee which doesn't fit any pattern.

\[
 \begin{array}{rcll}
    \mathcal{C} & = & \{ C_1^{k_1}, \dots, C_n^{k_n} \}\\
    \mathcal{V} & = & \mathcal{C}\,\mathcal{V}^*\\  
    \mathcal{P} & = & \_ \mid \mathcal{C}\,\mathcal{P}^*
 \end{array}
\]
\[
\setarrow{\xrightarrow}
\trans{\inbr{v;\,p_1,\dots,p_k}}{}{i},\quad 1\leqslant i\leqslant k; v \in \mathcal{V}; p_1,\dots,p_k \in \mathcal{P}
\]

For every synthesis task the patterns and indexes are ground. Type information is also available. For every subvalue in scrutinee we know which constructors makes sense to match, it's arities and type information of constructors' arguments.

The relation ``$\xrightarrow{}{}\!\!$'' gives us a \emph{declarative} semantics of pattern matching. Since we are interested in
synthesizing implementations, we need a \emph{programmatical} view on the same problem. Thus, we introduce a language $\mathcal S$
(the ``switch'' language) of test-and-branch constructs, and a evaluator ``$\xrightarrow{}{}_{\!\!\mathcal S}$'' that matches a scrutinee to an integer. 
In the original paper~\cite{Kosarev2020} this language has built-in $\primi{switch}$ construction that distinguishes tags (integers) of algebraic constructors. But for now we are only testing our approach on tuples of booleans, so we have only if-then-else construction in our language $\mathcal{S}$, despite that fact that $\primi{switch}$ supercedes $\primi{if-then-else}$.

\[
\begin{array}{rccl}
  \mathcal M & = &       & \bullet \\
             &   & \mid  & \mathcal M\,[\mathbb{N}] \\
  \ir        & = &       & \primi{return}\;\mathbb{N} \\
             &   & \mid  & \primi{switch}\;\mathcal{M}\;\primi{with}\; [\mathcal{C}\; \primi{\rightarrow}\; \ir]^*\;\primi{otherwise}\;\ir\\
             &   & \mid  & \primi{if}\;\mathcal{M}\;\primi{starts with}\;\mathcal{C}\;\primi{then}\; \ir\;\primi{else}\;\ir
\end{array}
\]

We can formulate the \emph{pattern matching synthesis problem} as follows: for a given ordered sequence of patterns $p_1,\dots,p_k$ find
a switch program $\withenv{v}{\pi}$, such that

\begin{center}
$\setarrow{\xrightarrow}
\forall v\in \mathcal V,\; \forall 1\leqslant i\leqslant n: \trans{\inbr{v;\,p_1,\dots,p_n}}{}{i}$\\
$\Longleftrightarrow$ \\ 
$\setsubarrow{_{\mathcal S}}\withenv{v}{\trans{\pi}{}{i}}$ \\
\end{center}

\noindent The description above uses universal quantification, and can't be immediately transformed into relational specification, because \emph{recursive} data types may make $\mathcal{V}$ infinite.
But there is another observation that makes this synthesis problem representable in \miniKanren{}.
For every synthesis task we have a ground set of patterns, and they check any scrutinee only into finite depth.
This allows one to cut the set of possible scrutinees until certain depth and replace universal quantification by a finite conjunction.
The downside of this encoding is an exponential blowup of search space:
\begin{itemize}
\item Increasing amount of constructors in types, increases amount of examples required, which hurts performance.
\item Increasing depth of constructors hurts performance, but it is expected.
\item Changing number of patterns while preserving the same maximum depth doesn't affect performance at all. This is unexpected.
\end{itemize}

%\todo{Now how wildcard variables could help us }. 
To reduce the number of required examples, we are going to use wildcard variables to say that scrutinee doesn't fit previous branches. In other words, every branch of pattern matching have a single corresponding example. 
That example will state via unification that scrutinee fits current branch, and also will state 
%using 
via disequality constraints with wildcards that scrutinee doesn't fit previous branches. 

During the search \OCanren{} accumulates inequalities between sub parts of scrutinee. 
More precisely, every $\primi{if-then-else}$ may introduce disequality between tags of algebraic constructors. It's rather easy to get into situation described in section~\ref{sec:intro}, where boolean constructor is not equal both \lstinline|true| and \lstinline|false|. 
To get rid of these bogus answers we enhance \OCanren{} by finite domain constraints using \Zthree{}~\cite{Zthree} under the hood. 
Adding this constraints complicates implementation of relational engine, because we can't no longer say two values are not equal because of their corresponding sub values are not equal (the inequality of sub values may contradict finite domain constraints). 
We speculate that in presence of finite domain constraints it may be better to store disequality constraints as DNF instead of CNF, but for now we use traditional\footnote{The recommended efficient way to represent disequality constraints is available online~\cite{fasterMK}} implementation.

%\todo{how we encode branches}

%\todo{domains}

Unfortunately, wildcards can't solve the pattern matching synthesis problem themselves in a way one can expect.
The result of pattern matching synthesis for program in Figure~\ref{fig:matching-example1} is shown in Figure~\ref{fig:matching-result-wierd}. We do believe that the expected reference answer corresponding to one from Figure~\ref{fig:matching-example2} will be eventually found by \OCanren{} but it is definitely far from being the first.
We will demonstrate the main source of such a behaviour by example below.

Consider pattern matching with the only branch \lstinline{__, false, true -> 1}.
The first answer produced by the interpreter on this example is \lstinline{q = 1}.
This is absolutely correct answer since on all scrutinees having \lstinline{false} as the second element and \lstinline{true} as the third element of a triple it produces 1.
In other words, the produced answer should have the same or bigger domain than the expected reference answer.
This can be seen as an analog of conservative approximation of the reference answer.
Further interpretation of other branches will make the answer produced on the previous step more precise but still conservative.

Processing the next example the interpreter will figure out \emph{some} scrutinee that satisfies disequality constraints and unifications from the given example.
Then it will change the program being synthesized to handle this particular scrutinee.
In other words, each interpreter step is some conservative approximation of the reference answer.
Thus, it is able to produce the reference answer iff on each step the approximation is precise.
This means that the reference answer should be eventually produced by the interpreter but after an unpredictable amount of time.

Finally, the interpreter ends up with a stream of answers (see Figure~\ref{fig:matching-result-wierd}) with each answer being in some sense a conservative approximation of the reference one.


%
% !TeX spellcheck = en_US
% !TEX root = main.tex

\section{Conclusion and future work}

%We presented an approach for pattern matching implementation synthesis using relational programming. Currently, it demonstrates a good performance only
%on a very small problems. The performance can be improved by searching for new ways to prune the search space and by speeding up the implementation of
%relations and structural constraints. Also it could be interesting to integrate structural constraints more closely into \textsc{OCanren}'s core.
%Discovering an optimal order of samples and reducing the complete set of samples is another direction for research.
%
%The language of intermediate representation can be altered, too. It is interesting to add to an intermediate language so-called \emph{exit nodes}
%described in~\cite{maranget2001}. The straightforward implementation of them might require nominal unification, but we are not aware of any
%\textsc{miniKanren} implementation in which both disequality constraints and nominal unification~\cite{alphaKanren} coexist nicely.
%
%At the moment we support only simple pattern matching without any extensions. It looks technically easy to extend our approach with
%non-linear and disjunctive patterns. It will, however, increase the search space and might require more optimizations.
%
%
%


\bibliography{references}
\appendix

% !TEX TS-program = pdflatex
% !TeX spellcheck = en_US
% !TEX root = main.tex

\section{Simple Wildcard Examples}
\label{appendix:examples}
%\textbf{TODO: INSERT A TABLE WITH EXAMPLES }

%\noindent

\begin{table}[H]
  \caption{A few example of relational queries involving wildcard variables}
  \small\centering

\begin{tabular}
    %{  p{4.5cm} | p{4cm}  }
    {c|c}
Example query  & Result \\ \hline
\lstinline|(__ =/= _.20)| &\lstinline|success| \\ \hline

\lstinline|(1, __) =/= (__, 1)| &\lstinline|fail| \\ \hline

\lstinline|(_.10, 2, __) =/= (1, __, 2)| &\lstinline|_.10 =/= 1| \\ \hline

\lstinline|(_.10, _.11) =/= (1, __)| &\lstinline|_.10 =/= 1| \\\hline
\begin{lstlisting}
fresh (a b)
  (q === pair a b)
  (q =/= pair !!1 __)
  (q === pair __ !!1)
\end{lstlisting} &\lstinline|q === (a [=/= 1], 1), b === 1| \\ \hline

%\lstinline|(__ =/= _.20)| &\lstinline|success| \\
\end{tabular}
\end{table}

\end{document}
