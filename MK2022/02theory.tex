% !TeX spellcheck = en_US
% !TEX root = main.tex

\section{Theory}
\label{sec:theory}

In this section we describe the essence of wildcard variables and how they interact with other features of \miniKanren{}.

\subsection{Wildcard in Disequality}
Checking and storing disequality constraints may be performed in clever manner~\cite{fasterMK}. But here we will describe na\"{i}ve implementation for simplicity. In standard implementation of disequality constraints are stored as a disjunction of conjunctions of pairs: a fresh variable and a term (or an another variable).


\subsection{Unification}
The primary usage of wildcard variables is in the context of disequality constraints. In the unification it is just a convenient syntax to avoid manual creation of fresh variable. 

\begin{minipage}{7cm}
\begin{lstlisting}{language=ocaml}
let tlo xs tl = 
  (xs === List.conso __ tl)
\end{lstlisting}
\end{minipage}
\begin{minipage}{9.5cm}
\begin{lstlisting}{language=ocaml}
let tlo xs tl = 
  fresh (h) (xs === List.conso h tl)
\end{lstlisting}
\end{minipage}
