% !TeX spellcheck = en_US
% !TEX root = main.tex
\section{Introduction}
\label{sec:intro}

Relational and logic programming are powerful techniques for enumerating the space of possible answers for a query. Various constraints allow us to prune search space and to make path to there right answer short. A few constraints (for example, disequality constraints~\cite{?}) are universally applicable, but it's OK to invent new special constraints for specific tasks. 

The \miniKanren{} family of languages includes different implementations, both statically and dynamically types. There are some peculiarities in statically typed implementation \OCanren{}~\cite{ocanren} relatively to ``official'' implementation~\cite{fasterMK}. For example, the following result of the query is decent in languages like 
\Scheme{}, but in statically typed \OCaml{} looks weird.

\begin{minipage}{7cm}
\begin{lstlisting}{language=scheme}
; Scheme
> (run 1 (q)  (=/= q #t) (=/= q #f) )
((_.0 (=/= ((_.0 #f)) ((_.0 #t)))))
\end{lstlisting}
\end{minipage}
\begin{minipage}{9.5cm}
\begin{lstlisting}{language=ocaml}
(* OCaml *)
> run ... (funq->(q =/= !!true) &&& (q =/= !!false))
q=_.0 [=/= false; =/= true];
\end{lstlisting}
\end{minipage}

\noindent Indeed, in \Scheme{} there are infinitely many possible values for variables that are neither \lstinline[]=#t=, nor 
\lstinline[]=#f=. But in \OCanren{} the compiler prohibits unification of variables with non-unifiable types, so variable \verb=q= could be bound in a substitution only to \lstinline=true=, \lstinline=false= or another fresh variable; and expected result of the query is empty stream.

The example above could be repaired by introduction of finite domain constraints~\cite{cKanren}, but in case of proper algebraic data types they doesn't help. Let's imagine that we want to express that \emph{a variable holds a list, but couldn't begin from a constructor \lstinline=Cons=}. The na\"{i}ve attempt in \OCanren{}, \lstinline|fresh (h tl) (q=/= cons h tl) | doesn't give us what we desire. It states that \empty{there are some} \lstinline|h| and \lstinline|tl| such that \lstinline|q| is not equal, but we expected that fact for any possible \lstinline|h| and \lstinline|tl|. This form of universal quantification is currently not expressible in \OCanren{}.

% TODO: write about expressivity in Scheme ?

\todo{Write something}

\begin{comment}
Algebraic data types (ADT) are an important tool in functional programming which deliver a way to represent flexible and easy to manipulate data structures.
To inspect the contents of an ADT's values a generic construct~--- \emph{pattern matching}~--- is used. Pattern matching can be considered as a generalization of
conventional conditional control-flow construct ``\lstinline|if .. then .. else|'' and in principle can be decomposed into a nested hierarchy of those; from
this standpoint the problem of pattern matching implementation can be considered trivial. However, some decompositions are obviously better than others. We
repeat here an example from~\cite{maranget2008} to demonstrate this difference (see Fig.~\ref{fig:match-example}). Here we match a triple of boolean
values $x$, $y$, and $z$ against four patterns (Fig.~\ref{fig:matching-example1}; we use \textsc{OCaml}~\cite{ocaml} as reference language). The na\"{i}ve
implementation of this example is shown on Fig.~\ref{fig:matching-example2}; however if we decide to match $y$ first the result becomes much
better (Fig.~\ref{fig:matching-example3}).

\begin{figure}[ht]
\begin{subfigure}[t]{0.2\linewidth}
\centering
\begin{lstlisting}
match x, y, z with
| _, F, T -> 1
| F, T, _ -> 2
| _, _, F -> 3
| _, _, T -> 4
\end{lstlisting}
\vskip18.5mm
\caption{Pattern matching}
\label{fig:matching-example1}
\end{subfigure}
\hspace{0.5cm}
\begin{subfigure}[t]{0.26\linewidth}
\centering
\begin{lstlisting}
if x then
  if y then
    if z then 4 else 3
  else
    if z then 1 else 3
else
  if y then 2
  else
    if z then 1 else 3
\end{lstlisting}
\caption{A correct but non-optimal\\\phantom{(b)~}implementation}
\label{fig:matching-example2}
\end{subfigure}
\hspace{0.5cm}
\begin{subfigure}[t]{0.33\linewidth}
\centering
\begin{lstlisting}
if y then
  if x then
    if z then 4 else 3
  else 2
else
  if z then 1 else 3
\end{lstlisting}
\vskip13.5mm
\caption{Optimal implementation}
\label{fig:matching-example3}
\end{subfigure}
\caption{Pattern matching implementation example} 
\label{fig:match-example}
\end{figure}

%clasification 1
%Although semantics of pattern matching can be given as a sequence of srutinee's sub expression comparisons (Figure~\ref{fig:matchpatts}) effective compilers don't follow
%this approach. 

The quality of a pattern matching implementation can be measured in various ways. One can either optimise the run time cost by minimizing the amount of checks performed, or the static
cost by minimizing the size of the generated code. \emph{Decision trees}
%~\cite{?} 
% I removed cite because 
% 1) can't find a good citation for it not in the field of compilers
% 2) folks in other papers don't refernece them. Only backtracking automata are begin referenced
are considered suitable for the first criterion as they check every subexpression no more than once.
However, minimizing the size of decision tree is known to be NP-hard~\cite{baudinet1985tree}, and as a rule various heuristics, using, for example,
the number of nodes, the length of the longest path and the average length of all paths are applied during compilation. In \cite{Scott2000WhenDM} the results of experimental
evaluation of nine heuristics for Standard ML of New Jersey are reported.

For minimizing the static cost \emph{backtracking automata} can be used since they admit a compact representation but in some cases can perform repeated checks.

%clasification 2
There is a certain difference in dealing with pattern matching in strict and non-strict languages. For strict languages checking sub-expressions of the scrutinee in any order is allowed.
The pattern matching implementation for strict languages can operate in \emph{direct} or \emph{indirect} styles. In the direct style the construction of an implementation is done explicitly. In indirect the construction of implementation requires some post-processing, which can vary from easy simplifications to complicated supercompilation
techniques~\cite{sestoft1996}. The main drawback of indirect approach is that the size of intermediate data structures can be exponentially large.

For non-strict languages pattern matching should evaluate only those sub-expressions which are necessary for performing pattern matching. If not done carefully pattern matching can
change the termination behavior of the program. In general non-strict languages put more constraints on pattern matching and thus admit a smaller  set of heuristics. 
A few approaches for checking sub-expressions in lazy languages have been proposed. In ~\cite{augustsson1985} a simple left-to-right order of subexpression checking was proposed
with a proof that this particular order doesn't affect termination. The backtracking automaton being built takes a form of a DAG to reduce the code size. A few refinements have been added in~\cite{wadler1987}
as a part of textbook~\cite{peytonjones1987} on the implementation of lazy functional languages. The approach from this book is used in the current version of GHC~\cite{marlow2012the}.
%~\footnote{The Glasgow Haskel Compiler, \url{http://www.haskell.org/ghc/}}
\cite{laville1991} models values in lazy languages using \emph{partial terms}, although it doesn't scale to types with infinite sets of constructors (like integers). The approach doesn't
test all subexpressions from left to right as does~\cite{augustsson1985} but aims to  avoid performing unnecessary checks by constructing \emph{lazy automaton}. Pattern matching for
lazy languages has been compiled also to decision trees~\cite{maranget1992} and later into \emph{decision DAGs} which in some cases allows the compiler to make the code
smaller~\cite{maranget1994}.

%about automata
The inefficiency of backtracking automata have been
improved in~\cite{maranget2001}. The approach utilizes a matrix representation for pattern matching. It splits the current matrix according to constructors in the
first column and reduces the task to compiling matrices with fewer rows. The technique is indirect; in the end a few optimizations are performed by introducing
special \emph{exit} nodes to the compiled representation. %No preprocessing is required for this scheme: or-pattern receives a special treatment during compilation process.
The approach from this paper is used in the current implementation of the \textsc{OCaml} compiler.

The previous approach uses the first column to split the matrix. In~\cite{maranget2008} the \emph{necessity} heuristic has been introduced which recommends which column should be
used to perform the split. Good decision trees which are constructed in this work can perform better in corner cases than~\cite{maranget2001}, but for practical use the
difference is insignificant.

While existing approaches deliver appropriate solutions for certain forms of pattern matching constructs, they have to be extended in an \emph{ad hoc} manner each time
the syntax and semantics of pattern matching construct changes. For example, besides a simple conventional form of pattern matching there are a number of extensions:
guards (first appeared in KRC language~\cite{turner2013}), disjunctive patterns~\cite{ocaml}, non-linear patterns~\cite{mcbride1969symbol}, active patterns~\cite{activepatterns} and pattern matching for polymorphic variants~\cite{Garrigue98} 
%and generalized algebraic datatypes~\cite{?}) 
which require a separate customized algorithms to be developed.

We present an approach to pattern matching implementation based on application of relational programming~\cite{TRS,WillThesis} and, in particular, relational interpreters~\cite{unified}
and relational conversion~\cite{conversion}. Our approach is based on relational representation of the source language pattern matching semantics on the one hand, and
the semantics of the intermediate-level implementation language on the other. We formulate the condition necessary for a correct and complete implementation of pattern matching and use it to
construct a top-level goal which represents a search procedure for all correct and complete implementations. We also present a number of techniques which make it possible to come up with an
\emph{optimal} solution as well as optimizations to improve the performance of the search. Similarly to many other prior works we use the size of the synthesized code, which can be measured statically, to distinguish better programs.
Our implementation\footnote{\url{https://github.com/kakadu/pat-match}} makes use of \textsc{OCanren}\footnote{\url{https://github.com/JetBrains-Research/OCanren}}~--- a typed
implementation of \textsc{miniKanren} for \textsc{OCaml}~\cite{OCanren}, and \textsc{noCanren}\footnote{\url{https://github.com/Lozov-Petr/noCanren}}~--- a converter from the subset
of plain \textsc{OCaml} into \textsc{OCanren}~\cite{conversion}. On an initial  evaluation, performed for a set of benchmarks taken from other papers, showed our synthesizer performing well.
However, being aware of some pitfalls of our approach, we came up with a set of counterexamples on which it did not provide any results in observable time, so we do not consider the problem
completely solved. We also started to work on mechanized formalization\footnote{\url{https://github.com/dboulytchev/Coq-matching-workout}}, written in \textsc{Coq}~\cite{Coq}, to
make the justification of our approach more solid and easier to verify, but this formalization is not yet complete. 

 
 
\end{comment}
