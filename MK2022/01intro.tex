% !TeX spellcheck = en_US
% !TEX root = main.tex
\section{Introduction}
\label{sec:intro}

Relational and logic programming are powerful techniques for enumerating the space of possible answers for a query. Constraints allow us to prune search space and to make path to there right answer short. Some constraints (for example, disequality~\cite{WillThesis}) are universally applicable, but it's OK to invent new special constraints for specific tasks. 

The \miniKanren{} family of languages includes different implementations, both statically and dynamically types. There are some peculiarities in statically typed implementation \OCanren{}~\cite{ocanren} relatively to ``official'' implementation~\cite{fasterMK}. For example, the following result of the query is decent in languages like 
\Scheme{}, but in statically typed \OCaml{} looks weird.

\begin{minipage}{7cm}
\begin{lstlisting}{language=scheme}
; Scheme
> (run 1 (q)  (=/= q #t) (=/= q #f) )
((_.0 (=/= ((_.0 #f)) ((_.0 #t)))))
\end{lstlisting}
\end{minipage}
\begin{minipage}{9.5cm}
\begin{lstlisting}{language=ocaml}
(* OCaml *)
> run ... (funq->(q =/= !!true) &&& (q =/= !!false))
q=_.0 [=/= false; =/= true];
\end{lstlisting}
\end{minipage}

\noindent Indeed, in \Scheme{} there are infinitely many possible values for variables that are neither \lstinline[]=#t=, nor 
\lstinline[]=#f=. But in \OCanren{} the compiler prohibits unification of variables with non-unifiable types, so variable \verb=q= could be bound in a substitution only to \lstinline=true=, \lstinline=false= or another fresh variable; and expected result of the query is empty stream.

The example above could be repaired by introduction of finite domain constraints~\cite{cKanren}, but in case of proper algebraic data types they doesn't help. Let's imagine that we want to express that \emph{a variable holds a list, but couldn't begin from a constructor \lstinline=Cons=}. The na\"{i}ve attempt in \OCanren{}, \lstinline|fresh (h tl) (q=/= cons h tl) | doesn't give us what we desire. It states that \empty{there are some} \lstinline|h| and \lstinline|tl| such that \lstinline|q| is not equal, but we expected that fact for any possible \lstinline|h| and \lstinline|tl|. This form of universal quantification is currently not expressible in \OCanren{}.

% TODO: write about expressivity in Scheme ?

\todo{Write something}
